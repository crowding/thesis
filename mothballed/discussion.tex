\documentclass[manuscript]{subfiles}

\begin{document}

\section{Discussion}

\todo[inline]{This section is not organized; more of a dumping ground for prompts.}

 In a field of dots undergoing random Brownian movement, a single dot that changes its position in a consistent direction is more easily detected than can be explained by local motion detectors situated along the object's trajectory and treated as independent samples \citep{Verghese:1999lq}. The enhancement in detectability seems to occur only after 100 ms of movement, possibly due to a process that responds to an initial cue from local motion detectors by reducing the number of detectors monitored to those in the vicinity of the initial motion signal, in particular those in the object's predicted path \citep{VerghesePreeti2002}. An interaction between local motion and position sensors thus appears necessary to account for performance at motion discrimination. \todo{MS: I expected to see a broader conclusion statement about synergy, cooperativity, or interaction... without returning the focus to the specific case involved. If you want to limit, as in some cooperive interaction ... at play... even in simple detection}

\todo[inline]{Discuss crowding as a source of uncertainty about (relative) position}

\todo[inline]{Discuss anatomical implications of eccentricity rule. Many brain areas are foveated. v4 receptive field, v1lateral connection size, other numerology}
 
\todo[inline]{why is it so hard to find global motion in the brain? We seem to locate it among a class of phenomena which integrate local visual features into higher level percepts. What about the ventral stream?}

Having two classes of stimuli that are well distinguished
behaviorally, we would hope to be able to find neurophysiological
correlates of both types of motion perception. The underpinnings of
local motion are \cite{Vaina:1996pi} provides one positive
finding, a patient with a unilateral cortical lesion slightly
posterior to the hMT+ complex, resulting in a deficit in detection of
global but not local motion in the contralateral field. Because
receptive fields in MT are large, reflecting the integration of many
V1 receptive fields spanning a range of spatial positions, it is
natural to suppose that a global motion process could be supported by
MT. However, attempts to observe MT neurons in the act of responding
to global, as opposed to local, motion, have met with little
success. For example, when random dot local motion stimuli are
presented in a window that moves independently of the dots, MT cells
respond primarily to to the local
motion \citep{Priebe:2001hl}. Responses in macaque MT and MST to a
stimulus opposing local and global motion showed no selectivity of
cell responses to global motion direction, even though the stimuli
elicited an oculomotor pursuit response in the direction of global
motion \citep{Ilg:2004qc}. \cite{Livingstone:2001ao} used sparse
noise to map second-order spatiotemporal kernels in MT receptive
fields and could not find any spatiotemporal interaction at scales
larger than those of V1 receptive fields. Finally, recordings in MT
made using stimuli similar to those used in this report find no
selectivity in MT cells for global
motion \citep{Shadlen:1993ne,Hedges:2004pr}. It appears that direction
selective responses in MT are, like those of V1, a function of local,
and not global, motion.

\cite{VerghesePreeti2002} found
that detectability of a single dot with consistent direction among
randomly moving backgrounds was enhanced after 100 ms of target
motion. In that report they also show (Figure 1 of that paper) that
there is some enhancement is preserved even if the target motion is
discontinuous; a target that suddenly jumped sideways or backwards in
the middle of its motion trajectory still had enhanced detectability
relative to the baseline. However, if the size of the jump was too
large, the enhancement vanished vanished. In the light of the present
study, we suggest that the critical maximum jump size may be the same
as the critical spacing of crowding. The enhancement observed in that
paper has been interpreted as a process that responds to an initial
cue from local motion detectors by reducing the number of detectors
monitored to just those in the vicinity of the initial motion
signal. It may be that the mechanism that winnows the pool of motion
detectors is the same as the integration field of \cite{Pelli:2004km}.

\biblio
\end{document}