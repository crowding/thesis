\documentclass[manuscript]{subfiles}

\begin{document}

\section[Introduction]{Introduction}\label{sec:introduction}

Accurate motion perception is essential for visually guided movement; complex behaviors such as chasing prey or catching a thrown ball require that an organism be able to rapidly determine the position and velocity of a moving object, and to anticipate its trajectory through space. 
 
Psychological research on motion perception has established that local and global motion are processed by separate mechanisms, an idea that dates back to Wertheimer's phenomenological distinction between fast `phi' and slower `beta' motion \citep{Steinman:2000ap}. It is believed that local (or short range) mechanisms respond to the motion of visual features defined by luminance, or by motion energy in the Fourier domain \citep{Adelson:1985ea}, at short temporal and spatial scales. Long-range or global motion \todo{I have a distaste for `global' as a name for the second process. It leads people down a wrong path of thinking about the wheel stimulus as having a `global' movement i.e. globally consistent with solid body rotation of the wheel, whereas what we are \em{actually} talking about here is a perception of the \emph{shift in position} of different elements; it's quite easy to have a perception of the wheel stimulus that is \emph{not} coherent with a solid body.} does not necessarily require features to differ in mean luminance from the background, does not necessarily contain motion energy in the Fourier domain, and is characterized by operating over longer spatial and temporal scales than the short-range process. Strong global motion percepts can be elicited by stimuli that modulate in contrast or texture rather than in luminance. In general, global motion seems to involve the change in position, over time, of some feature in the image. \citep{Lu:1995la}.

It is thought that local motion mechanism originates in cells in V1. Classic models of local motion include local filters that extract directional signals from space-time correlations in luminance contrast in a small region of space, in a way similar to what has been observed to drive the activity of cells in V1 \todo{cites:(Reichardt; Adelson; Simoncelli; Ullman \& Marr; others)}. Psychophysical data are consistent with the notion that the size of the local motion filters are similar in size of the classical receptive field of a V1 neuron. \todo{cites} Global motion on the other hand consists of processes that track an object over distances larger than what can be achieved through individual local filters. A strong sensation of global motion can be created without any local motion signal, so it is possible that local and global mechanisms have independent origins. The anatomical substrate of global motion mechanism is unknown. While MT and MST have been proposed as candidates, \emph{certain properties of these regions don't match what is known about our sensations of global motion. (Movshon, etc.)}

One thing that makes accurate estimation of motion particularly difficult is that the local motion signals are not always a reliable indication of the global motion. In a complicated visual world, motion can come from many sources, and accurate perception of the movement of objects requires disambiguating those motion signals attributable to the object from irrelevant motions in the background or of other objects. Take the example of a frog trying to catch a zebra butterfly within waving grass. While many of the local signals will correctly signal the motion of the butterfly, the motion of the grass creates a subset of local motion signals that are substantially incorrect. Thus, a simple average of the motion over any receptive field large enough to cover more than a tiny portion of the scene will not accomplish the goal of tracking the motion of the butterfly. On the other hand, an average of motion signals within very small regions will also be incorrect; small regions containing only one edge cannot uniquely determine the motion of an object (the `aperture problem', etc.) To track the butterfly, then requires integrating local motion signals over space and time in a way that is consistent with the global change in position of an object, while discarding local signals that are inconsistent with the object's trajectory.

In this paper we begin to examine how local motion and global changes in position interact in forming an overall perception of motion. We constructed stimuli that combined a local motion with global position shift; such that the local and the global motion can either be congruent or incongruent.\autoref{mov:wheels} shows an example of how local and global motion signals can interact. 

The movie shows two wheels, each composed of five moving elements. The elements are composed of a succession of brief wavelet pulses, at $\unit[100]{ms}$ intervals. These pulses have the same mean luminance as the background, and the envelope of each pulse does not move, however, the peaks and troughs of the wavelet move within the envelope during the pulse, giving the \emph{local} component of the motion stimulus. The \emph{global} motion component of the stimulus is created by offsetting the position of each pulse relative to the last. In this way we can control local and global motion of an object largely independently On the right side of the display, the local and global motion components move in the same direction; on the left side the local and global components are in opposite directions. Full details of the construction of this display are given in \autoref{sec:stimuli}.

\begin{movie}
  \includemovie[poster,repeat,text={\small(Movie: counter-rotating wheels)}]{6in}{3in}{demo_counter.mov}
  \caption{When fixating at the center of the left wheel, both wheels appear to move in the same direction. But when fixating the center of the right wheel, both wheels appear to move in opposite directions. The appearance of the right wheel's movement reverses depending on the viewing eccentricity.}
  \label{mov:wheels}
\end{movie}

This display elicits an illusion of motion reversal. The rotation of the spots around the fixation point on the right side of the display appears constant regardless of viewing angle. In contrast, the motion of the spots on the left side appears to change direction based on where one fixates. When fixating in the center of the left circle, the spots appear to travel counterclockwise around the circle (consistent with global motion); when viewed parafoveally, the spots appear to move clockwise (consistent with local motion). When making an eye movement that shifts the right circle from a parafoveal to a foveal location, or vice versa, it appears to suddenly reverse its direction.

Because every element in our display contains both local and global motion, it was natural to ask whether the eccentricity-driven reversal of apparent motion direction happened with only one spot moving around the fixation point. It did not; when all but one of the spots in the circle were eliminated, the remaining spot appeared to move consistent with its global position shift, regardless of eccentricity. This suggests that interactions between the elements play an important role in causing the local motion to dominate when viewing the stimulus parafoveally. In other words, under conditions of crowding the local percept dominates. 

Crowding is a phenomenon wherein identification or discrimination of an object presented in the visual periphery is impaired by the presence of nearby, but non-overlapping flanking objects. A finding characteristic of crowding is that critical spacing (usually a measure of the distance between target and flanker which achieves a particular elevation of threshold for recognition) scales linearly with retinal eccentricity \citep{Bouma:1970ng,Toet:1992db}. Although most studies of crowding focus on its effect of impairing the recognition of shapes (e.g. letters) in parafoveal vision, it has become apparent that crowding is a more general phenomenon, extending to many different types of visual features (e.g. (\citealt{Berg:2007rc}; for review, see \citealt{Levi:2008la}) It is thought that crowding is characteristic of some cortical mechanism that integrates signals from low-level feature detectors, a so-called ``integration field'' \citep{Pelli:2004km}. Because the scaling of critical distance with spacing mirrors the variation of cortical magnification with eccentricity, the integration field is thought to be a process that subsumes a constant distance on the cortical surface \citep{Pelli:2008nx}.

\citet{Pelli:2004km} proposed that the crucial diagnostic test for crowding as opposed to masking or other forms of spatial interference is that the critical spacing scales with eccentricity and is relatively unaffected by signal size. Accordingly, we set out to determine which target spacing and motion parameters are necessary to drive the reversal of apparent motion as various eccentricities. In \autoref{sec:constant} below, we determine the relationship between critical spacing and target spacing, which satisfies Bouma's law. In \autoref{sec:occlusion} we show that the critical spacing is unaffected by the presence of an occluder which covers 2/3 of the visible circle, meaning that it is the spacing which is relevant and not the number of visible targets. In \autoref{sec:grid} we show that the critical distance and scaling property is robust to the size of the stimuli. We also test its robustness to variations in temporal frequency, step size, and step interval. 

While most studies of crowding involve stationary stimuli, motion stimuli add a temporal component. In Experiments 1 through 3 we consistently find that for stimuli near the crowding distance, the trials for which the subject took longer in responding were more likely to correctly reflect the global direction of motion. In Experiment 4 we use an auditory cue to vary the subjects' response time to investigate this effect in more detail. Our results reinforce the idea that global motion processing is the result of an integration of the output of low-level feature detectors, and that in fact the process subserving detection of global motion might be identical to the processes underlying object recognition and target selection. We discuss the implications for possible mechanisms of higher order motion perception and speculate on their possible physiological implementations.

\biblio
\end{document}
