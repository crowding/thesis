\documentclass[manuscript]{subfiles}

\begin{document}

\section[Introduction]{Introduction}\label{sec:introduction}

The ability to detect moving objects is an essential function of the visual system. Accurate motion perception is essential for visually guided movement; complex behaviors such as chasing prey or catching a thrown ball require that an organism be able to rapidly determine the position and velocity of a moving object, and to anticipate its trajectory through space.

Psychological research on motion perception has established that separate mechanisms are involved, an idea that dates back to Wertheimer's phenomenological distinction between fast `phi' and slower `beta' motion \citep{Steinman:2000ap}. One mechanism responds to the motion of visual features that are defined by luminance, or by motion energy in the Fourier domain \citep{Adelson:1985ea}, at short temporal and spatial scales. These stimuli contain what is variously known as local, first-order or short range motion. Another class of stimuli (long-range, higher-order or global motion) results in a perception of motion without requiring features to differ in mean luminance from the background, and without containing motion energy in the Fourier domain. Some examples of global motion stimuli are contrast modulation or texture flicker stimuli; the forms of global motion are varied, but in general involve the change in position, over time, of some feature in the image. \citep{Lu:1995la}.

\todo[inline]{MS suggests something adding something about what is known of the mechanisms here as well as the stimuli. ``we need to set up the idea that the variety of motion systems can be thought of in these terms. One process is comprised of local nonlinear filters that extract directional signals from space-time correlations in luminance contrast in a small region of space (Reichardt; Adelson; Simoncelli; Ullman \& Marr; others). Other processes are best viewed as tracking a corresponding object over distances larger than what can be achieved through these filters. We can admit up front that the size of the filters in the first process is not known, but it might be associated with the size of a classic RF of a V1 neuron, or a fraction of this (e.g., <180 deg phase shift with respect to its characteristic period). This definition conforms at least roughly to the distinction between short and long range motion systems.''}

Objects moving realistically change position as well as possess local motion energy, so that they activate all motion systems concurrently in varying ways. Although many properties of local and global motion systems have been elucidated by using stimuli designed to isolate the response of one system without driving the other, it may be the case that local and global mechanisms interact in order to exploit the natural connection between motion energy and change in position. In that case the interaction between local and global motion mechanisms will be key to understanding motion perception as a whole. Indeed there is evidence that a system responding to a consistent change in position of a stimulus interacts with the output of local motion detectors. In a field of dots undergoing random Brownian movement, a single dot that changes its position in a consistent direction is more easily detected than can be explained by local motion detectors situated along the object's trajectory and treated as independent samples \citep{Verghese:1999lq}. The enhancement in detectability seems to occur only after 100 ms of movement, possibly due to a process that responds to an initial cue from local motion detectors by reducing the number of detectors monitored to those in the vicinity of the initial motion signal, in particular those in the object's predicted path \citep{VerghesePreeti2002}. An interaction between local motion and position sensors thus appears necessary to account for performance at motion discrimination. \todo{MS: I expected to see a broader conclusion statement about synergy, cooperativity, or interaction... without returning the focus to the specific case involved. If you want to limit, as in some cooperive interaction ... at play... even in simple detection}

To investigate the interaction between local motion and position signals, we constructed stimuli that combined a local motion with global position shift. By setting the local motion and position shift in opposition, we produced a striking illusion of motion reversal. Here (\autoref{mov:wheels}) we show a display with two wheels each containing five moving elements. The elements are compused of a succession of brief wavelet pulses, at $\unit[100]{ms}$ intervals. These pulses have the same mean luminance as the background and the envelope of each pulse does not move, however, the peaks and troughs of the wavelet move within the envelope during the pulse, giving the local component of the motion stimulus. To give a global motion component to the stimulus, the position of each pulse is offset relative to the last. Thus local and global components can be changed independently. On the left side of the display, the local and global motion components are in the same direction; on the right side the local and global components are in opposite directions. Full details of the construction of this display are given in \autoref{sec:stimuli}.

\begin{movie}
  \includemovie[poster,repeat,text={\small(Movie: counter-rotating wheels)}]{6in}{3in}{demo_counter.mov}
  \caption{When fixating at the center of the left wheel, both wheels appear to move in the same direction. But when fixating the center of the right wheel, both wheels appear to move in opposite directions. The appearance of the right wheel's movement reverses depending on the viewing eccentricity.}
  \label{mov:wheels}
\end{movie}

In this display the rotation of the spots around the fixation point on the left side of the display appears constant regardless of viewing angle, but the motion of the spots in the right side appears to change direction based on retinal eccentricity. When fixating in the center of the right circle, the spots appear to travel clockwise around the circle; when viewed parafoveally, the spots appear to move counterclockwise. When making an eye movement that shifts the right circle from a parafoveal to a foveal location, or vice versa, it appears to suddenly reverse its direction.

\todo{Most of this para moved to Discussion.} An eccentricity-dependent reversal in perceived direction has been previously reported for some reverse-phi stimuli \citep{Mather:1985rt,Chubb:1989fj}; a display similar to ours, arranging discrete elements in circles, has also been independently developed by \citet{Shapiro:2008ek}. \citet{Chubb:1989fj} proposed that the global motion in reverse-phi displays was detected by rectification of the output of some feature detectors, after which a more normal motion-energy filtering process followed, and that the resolution of this rectification and detection was weaker in the periphery. Because our display assigns local and global motion to the same features, it was natural to ask whether the eccentricity-driven reversal of apparent motion direction happened with only one spot moving around the fixation point. It did not; when all but one of the spots in the circle were eliminated, the remaining spot appeared to move consistent with its global position shift, regardless of eccentricity. Because the targets were the same size in both conditions, a motion energy detector operating on rectified input should have performed as well in both cases. Instead this illusion suggested that a long range interference between the distinct elements was responsible for the failure of global apparent motion in parafoveal viewing. In other words, a form of crowding limits the detection of shifts in global position.

Crowding is a phenomenon wherein identification or discrimination of an object presented in the visual periphery is impaired by the presence of nearby, but non-overlapping flanking objects. A finding characteristic of crowding is that critical spacing (usually a measure of the distance between target and flanker which achieves a particular elevation of threshold for recognition) scales linearly with retinal eccentricity \citep{Bouma:1970ng,Toet:1992db}. Although most studies of crowding focus on its effect of impairing the recognition of shapes (e.g. letters) in parafoveal vision, it has become apparent that crowding is a more general phenomenon, extending to many different types of visual features (e.g. (\citealt{Berg:2007rc}; for review, see \citealt{Levi:2008la}) It is thought that crowding is characteristic of some cortical mechanism that integrates signals from low-level feature detectors, a so-called ``integration field'' \citep{Pelli:2004km}. Because the scaling of critical distance with spacing mirrors the variation of cortical magnification with eccentricity, the integration field is thought to be a process that subsumes a constant distance on the cortical surface \citep{Pelli:2008nx}.

\todo[inline]{include the crowding demo or the single-element wheeldemo here?}

\citet{Pelli:2004km} proposed that the crucial diagnostic test for crowding as opposed to masking or other forms of spatial interference is that the critical spacing scales with eccentricity and is relatively unaffected by signal size. Accordingly, we set out to determine which target spacing and motion parameters are necessary to drive the reversal of apparent motion as various eccentricities. In Experiment 1 below, we determine the relationship between critical spacing and target spacing, which satisfies Bouma's law. In Experiment 2 we show that the critical distance and scaling property is robust to the size of the stimuli. We also test its robustness to variations in temporal frequency, step size, and step interval. In Experiment 3 we show that the critical spacing is unaffected by the presence of an occluder which covers 2/3 of the visible circle, meaning that it is the spacing which is relevant and not the number of visible targets.

While most studies of crowding involve stationary stimuli, motion stimuli add a temporal component. In Experiments 1 through 3 we consistently find that for stimuli near the crowding distance, the trials for which the subject took longer in responding were more likely to correctly reflect the global direction of motion. In Experiment 4 we use an auditory cue to vary the subjects' response time to investigate this effect in more detail. Our results reinforce the idea that global motion processing is the result of an integration of the output of low-level feature detectors, and that in fact the process subserving detection of global motion might be identical to the processes underlying object recognition and target selection. We discuss the implications for possible mechanisms of higher order motion perception and speculate on their possible physiological implementations.

\biblio
\end{document}
