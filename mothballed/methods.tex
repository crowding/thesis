\documentclass[manuscript]{subfiles}

\begin{document}

\section{General Methods}\label{sec:methods}

\subsection{Subjects}
Five subjects took part in this series of experiments. The subjects and the experiments they took part in are listed in Table 1. Subject P.M. is an author. Subject S.K. was made aware of the purpose of the experiments only after completing \autoref{sec:constant}. Subjects S.M., D.T., and G.B. were paid and were naive to the purpose of the experiments. 

\subsection{Equipment}

Stimuli were presented on a flat CRT video monitor (ViewSonic PF790; $800 \times 600$ pixels; display area $341 \times 256$ mm; 120Hz refresh rate) Experiments were programmed in MATLAB using the Psychtoolbox \citep{Brainard:1997gq} and Eyelink toolbox extensions \citep{Cornelissen:2002wl}, along with custom OpenGL code. All stimuli were presented on a 50\% gray background whose luminance was $33.10 cd/m^2$. The display had a black level of $0.10 cd/m^2$ and a white of $66.05 cd/m^2$ measured against the gray background. \todo{It would be cute to pull this out from data but I have more pressing concerns.}

Subjects sat behind a blackout curtain so that ambient illumination
was mostly due to the monitor and viewed the screen binocularly using
a chin and forehead rest with the eyes 60 cm from the screen. Eye
position was monitored using a video-based eye tracker (EyeLink 1000;
SR Research) using a sample rate of 250 Hz. Eye movements were
recorded but are not reported in this paper. Subjects gave responses
by turning a knob (PowerMate; Griffin Technologies) with their
preferred hand.

\subsection{Stimuli}\label{sec:stimuli}

Example stimuli are shown in \autoref{mov:stimuli} and are illustrated in an $(x, t)$ plot in \autoref{fig:stimuli} ($x$ here being a slice around a circle centered on the fixation point and passing through the center of each motion element.) The stimuli consisted of discrete local motion elements
presented at regular temporal and spatial intervals as in apparent
motion. Each local motion element had a luminance profile along the
circle given by a Cauchy filter function
\citep{Klein:1985rz} with peak spatial frequency $f$. The luminance
profile shifts phase with a constant temporal frequency $\omega$ and
is temporally modulated by a Gaussian envelope with standard deviation
$d/2$. In the radial direction, each local motion
element had a Gaussian envelope with standard deviation $w/2$. The
equation describing the luminance profile of a patch as a function of
position and time is then:
\begin{equation*}
C(x, y, t) = \mathrm{cos}^n(\mathrm{tan}^{-1}(fx/n))\mathrm{cos}(n \cdot \mathrm{tan}^{-1}(fx/n) + {\omega}t) e^{-(t/2d)^2-(y/2d)^2}
\end{equation*}
with the direction of motion along $x$. \todo{Is the equation totally necessary? Is the use of x,y to describe a circular stimulus confusing here?} The spatial bandwidth
parameter $n$ was set to 4 for all stimuli.

\begin{movie}
  \includemovie[poster,text={\small(Movie: three example stimuli)}]{4in}{4in}{demo_stimuli.mov}
  \caption{Three example stimuli. Subjects viewed stimuli such as these and were asked to judge the direction ofmovement of the elements (here clockwise in all cases) regardless of the direction of local motion (clockwise, neutral counterphase, or counterclockwise)}
  \label{mov:stimuli}
\end{movie}

In each trial, a number of identical elements were arranged in a circle around the fixation point, each oriented with the direction of motion tangential to the circle. Each element was presented repeatedly at intervals of ${\Delta}t$, each successive appearance displaced a fixed distance ${\Delta}x$ around the circle. The examples in \autoref{mov:stimuli} have the following settings, the same as used in Experiment 1 \todo{Double check these figures}: ${\Delta}t = 100$ ms, $\omega = 10$ cyc/s, $d$ = 0.033 s, and if $\phi$ denotes eccentricity, then $f = 8.9 cyc/\phi$, ${\Delta}x = 0.05 \cdot \phi$, and $w = 0.066 \cdot \phi$. The contrast of the local motion elements was 100\% for trials using counterphase stimuli, and $70.7\%$ for other trials (so as to keep the motion-energy of the display constant.)

For Experiments 1, 2, and 3, subjects were required to respond within a fixed temporal window. If the latency from motion onset to response was outside the window, the fixation point changed color (red for late responses, blue for early responses) for 1 second as feedback and the trial was reshuffled into the stimulus set to be repeated later in the session.

%The response window was chosen for each subject based on %preliminary sessions and is listed in Table 1. I think I'm using a constant response window for everyone, it's only SM who was superfast and his data is no good anyway.

Subjects performed the task in sessions of at most 1 hour, divided into 4 or 5 blocks of 150 to 200 trials each, and were prompted to take a break between blocks. Subjects could also rest at any point by simply delaying fixation. At the beginning of each block, the eye tracking system was automatically recalibrated by asking the subject to make saccades to a sequence of targets at randomly chosen locations on the screen.

For all experiments reported here, three stimulus types were used with equal probability. In one third of trials the direction of local motion was congruent with that of global motion. In the second third, the direction of local motion was opposite to the direction of global motion. In the remaining trials, elements with counterphase local motion were used. Counterphase elements were constructed by superposing two local motion elements with equal and opposite directions of local motion.; i.e. the counterphase stimuli have the same spatial and temporal frequency content as the congruent and incongruent elements, but their motion energy is equivocal between opposite directions. The second stimulus in \autoref{mov:stimuli} shows counterphase local motion.

\biblio
\end{document}