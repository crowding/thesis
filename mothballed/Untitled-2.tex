\documentclass[../manuscript.tex]{subfiles}
\begin{document}

A noted property of crowding in parafoveal vision is that the range of
spatial interaction between nearby targets is not dependent on the
size of the stimuli \cite{Levi:2002cs}. To determine whither the
motion reversal illusion shared this property we collected thresholds
for each eccentricity under altered stimulus configurations. In
separate sessions we varied spatial frequency (using values of $\phi$
scaled by 66\%, 100\% and 150\% compared to \autoref{sec:constant}), temporal
frequency (using values of 6.6, 10, and 15 Hz), spatial step size
(scaling values of ${\Delta}x$ by 66\%, 100\% and 150\% compared to
\autoref{sec:constant}), and temporal step interval (using ${\Delta}t$ values of
66, 100, and 150 ms, and $d$ values of 44, 66, and 100 ms,
respectively). For these experiments we used the QUEST procedure
\cite{Watson:1983hc} to select the the number of targets at each
trial. As before, there were three types of trials, with local motion
congruent, incongruent, or ambivalent to the global translation. The
QUEST algorithm selected the target spacing for all trial types, but
the subject's response was used to update the QUEST estimate only for
trials with local motion incongruent to global. Separare, randomly
interleaved estimations were performed for each stimulus configuration
and eccentricity. The QUEST algorithm was only used to efficiently
select stimulus values and not to obtain final threshold estimates; we
re-fit the data using maximum likelihood logistic regression, as in
\autoref{sec:constant}.

\begin{figure}
  \caption{Effect of target properties on critical spacing. A. Each
    plot shows critical spacing versus eccentricity for three
    different stimulus conditions. in one subject. B. Normalized
    logistic regression coefficient for each stimulus condition in
    each subject. * indicates significance ($p < 0.5$) }
\end{figure}

\end{document}