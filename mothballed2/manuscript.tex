\batchmode
\documentclass[english,jou]{article}
\newcommand{\hlnumber}[1]{\textcolor[rgb]{0.0823529411764706,0.0784313725490196,0.709803921568627}{#1}}%
\newcommand{\hlfunctioncall}[1]{\textcolor[rgb]{1,0,0}{#1}}%
\newcommand{\hlstring}[1]{\textcolor[rgb]{0.6,0.6,1}{#1}}%
\newcommand{\hlkeyword}[1]{\textcolor[rgb]{0,0,0}{\textbf{#1}}}%
\newcommand{\hlargument}[1]{\textcolor[rgb]{0.694117647058824,0.247058823529412,0.0196078431372549}{#1}}%
\newcommand{\hlcomment}[1]{\textcolor[rgb]{0.8,0.8,0.8}{#1}}%
\newcommand{\hlroxygencomment}[1]{\textcolor[rgb]{0,0.592156862745098,1}{#1}}%
\newcommand{\hlformalargs}[1]{\textcolor[rgb]{0.0705882352941176,0.713725490196078,0.0705882352941176}{#1}}%
\newcommand{\hleqformalargs}[1]{\textcolor[rgb]{0.0705882352941176,0.713725490196078,0.0705882352941176}{#1}}%
\newcommand{\hlassignement}[1]{\textcolor[rgb]{0.215686274509804,0.215686274509804,0.384313725490196}{\textbf{#1}}}%
\newcommand{\hlpackage}[1]{\textcolor[rgb]{0.588235294117647,0.713725490196078,0.145098039215686}{#1}}%
\newcommand{\hlslot}[1]{\textit{#1}}%
\newcommand{\hlsymbol}[1]{\textcolor[rgb]{0,0,0}{#1}}%
\newcommand{\hlprompt}[1]{\textcolor[rgb]{0,0,0}{#1}}%

\usepackage{color}%
 
\newsavebox{\hlnormalsizeboxclosebrace}%
\newsavebox{\hlnormalsizeboxopenbrace}%
\newsavebox{\hlnormalsizeboxbackslash}%
\newsavebox{\hlnormalsizeboxlessthan}%
\newsavebox{\hlnormalsizeboxgreaterthan}%
\newsavebox{\hlnormalsizeboxdollar}%
\newsavebox{\hlnormalsizeboxunderscore}%
\newsavebox{\hlnormalsizeboxand}%
\newsavebox{\hlnormalsizeboxhash}%
\newsavebox{\hlnormalsizeboxat}%
\newsavebox{\hlnormalsizeboxpercent}% 
\newsavebox{\hlnormalsizeboxhat}%
\newsavebox{\hlnormalsizeboxsinglequote}%
\newsavebox{\hlnormalsizeboxbacktick}%

\setbox\hlnormalsizeboxopenbrace=\hbox{\begin{normalsize}\verb.{.\end{normalsize}}%
\setbox\hlnormalsizeboxclosebrace=\hbox{\begin{normalsize}\verb.}.\end{normalsize}}%
\setbox\hlnormalsizeboxlessthan=\hbox{\begin{normalsize}\verb.<.\end{normalsize}}%
\setbox\hlnormalsizeboxdollar=\hbox{\begin{normalsize}\verb.$.\end{normalsize}}%
\setbox\hlnormalsizeboxunderscore=\hbox{\begin{normalsize}\verb._.\end{normalsize}}%
\setbox\hlnormalsizeboxand=\hbox{\begin{normalsize}\verb.&.\end{normalsize}}%
\setbox\hlnormalsizeboxhash=\hbox{\begin{normalsize}\verb.#.\end{normalsize}}%
\setbox\hlnormalsizeboxat=\hbox{\begin{normalsize}\verb.@.\end{normalsize}}%
\setbox\hlnormalsizeboxbackslash=\hbox{\begin{normalsize}\verb.\.\end{normalsize}}%
\setbox\hlnormalsizeboxgreaterthan=\hbox{\begin{normalsize}\verb.>.\end{normalsize}}%
\setbox\hlnormalsizeboxpercent=\hbox{\begin{normalsize}\verb.%.\end{normalsize}}%
\setbox\hlnormalsizeboxhat=\hbox{\begin{normalsize}\verb.^.\end{normalsize}}%
\setbox\hlnormalsizeboxsinglequote=\hbox{\begin{normalsize}\verb.'.\end{normalsize}}%
\setbox\hlnormalsizeboxbacktick=\hbox{\begin{normalsize}\verb.`.\end{normalsize}}%
\setbox\hlnormalsizeboxhat=\hbox{\begin{normalsize}\verb.^.\end{normalsize}}%



\newsavebox{\hltinyboxclosebrace}%
\newsavebox{\hltinyboxopenbrace}%
\newsavebox{\hltinyboxbackslash}%
\newsavebox{\hltinyboxlessthan}%
\newsavebox{\hltinyboxgreaterthan}%
\newsavebox{\hltinyboxdollar}%
\newsavebox{\hltinyboxunderscore}%
\newsavebox{\hltinyboxand}%
\newsavebox{\hltinyboxhash}%
\newsavebox{\hltinyboxat}%
\newsavebox{\hltinyboxpercent}% 
\newsavebox{\hltinyboxhat}%
\newsavebox{\hltinyboxsinglequote}%
\newsavebox{\hltinyboxbacktick}%

\setbox\hltinyboxopenbrace=\hbox{\begin{tiny}\verb.{.\end{tiny}}%
\setbox\hltinyboxclosebrace=\hbox{\begin{tiny}\verb.}.\end{tiny}}%
\setbox\hltinyboxlessthan=\hbox{\begin{tiny}\verb.<.\end{tiny}}%
\setbox\hltinyboxdollar=\hbox{\begin{tiny}\verb.$.\end{tiny}}%
\setbox\hltinyboxunderscore=\hbox{\begin{tiny}\verb._.\end{tiny}}%
\setbox\hltinyboxand=\hbox{\begin{tiny}\verb.&.\end{tiny}}%
\setbox\hltinyboxhash=\hbox{\begin{tiny}\verb.#.\end{tiny}}%
\setbox\hltinyboxat=\hbox{\begin{tiny}\verb.@.\end{tiny}}%
\setbox\hltinyboxbackslash=\hbox{\begin{tiny}\verb.\.\end{tiny}}%
\setbox\hltinyboxgreaterthan=\hbox{\begin{tiny}\verb.>.\end{tiny}}%
\setbox\hltinyboxpercent=\hbox{\begin{tiny}\verb.%.\end{tiny}}%
\setbox\hltinyboxhat=\hbox{\begin{tiny}\verb.^.\end{tiny}}%
\setbox\hltinyboxsinglequote=\hbox{\begin{tiny}\verb.'.\end{tiny}}%
\setbox\hltinyboxbacktick=\hbox{\begin{tiny}\verb.`.\end{tiny}}%
\setbox\hltinyboxhat=\hbox{\begin{tiny}\verb.^.\end{tiny}}%



\newsavebox{\hlscriptsizeboxclosebrace}%
\newsavebox{\hlscriptsizeboxopenbrace}%
\newsavebox{\hlscriptsizeboxbackslash}%
\newsavebox{\hlscriptsizeboxlessthan}%
\newsavebox{\hlscriptsizeboxgreaterthan}%
\newsavebox{\hlscriptsizeboxdollar}%
\newsavebox{\hlscriptsizeboxunderscore}%
\newsavebox{\hlscriptsizeboxand}%
\newsavebox{\hlscriptsizeboxhash}%
\newsavebox{\hlscriptsizeboxat}%
\newsavebox{\hlscriptsizeboxpercent}% 
\newsavebox{\hlscriptsizeboxhat}%
\newsavebox{\hlscriptsizeboxsinglequote}%
\newsavebox{\hlscriptsizeboxbacktick}%

\setbox\hlscriptsizeboxopenbrace=\hbox{\begin{scriptsize}\verb.{.\end{scriptsize}}%
\setbox\hlscriptsizeboxclosebrace=\hbox{\begin{scriptsize}\verb.}.\end{scriptsize}}%
\setbox\hlscriptsizeboxlessthan=\hbox{\begin{scriptsize}\verb.<.\end{scriptsize}}%
\setbox\hlscriptsizeboxdollar=\hbox{\begin{scriptsize}\verb.$.\end{scriptsize}}%
\setbox\hlscriptsizeboxunderscore=\hbox{\begin{scriptsize}\verb._.\end{scriptsize}}%
\setbox\hlscriptsizeboxand=\hbox{\begin{scriptsize}\verb.&.\end{scriptsize}}%
\setbox\hlscriptsizeboxhash=\hbox{\begin{scriptsize}\verb.#.\end{scriptsize}}%
\setbox\hlscriptsizeboxat=\hbox{\begin{scriptsize}\verb.@.\end{scriptsize}}%
\setbox\hlscriptsizeboxbackslash=\hbox{\begin{scriptsize}\verb.\.\end{scriptsize}}%
\setbox\hlscriptsizeboxgreaterthan=\hbox{\begin{scriptsize}\verb.>.\end{scriptsize}}%
\setbox\hlscriptsizeboxpercent=\hbox{\begin{scriptsize}\verb.%.\end{scriptsize}}%
\setbox\hlscriptsizeboxhat=\hbox{\begin{scriptsize}\verb.^.\end{scriptsize}}%
\setbox\hlscriptsizeboxsinglequote=\hbox{\begin{scriptsize}\verb.'.\end{scriptsize}}%
\setbox\hlscriptsizeboxbacktick=\hbox{\begin{scriptsize}\verb.`.\end{scriptsize}}%
\setbox\hlscriptsizeboxhat=\hbox{\begin{scriptsize}\verb.^.\end{scriptsize}}%



\newsavebox{\hlfootnotesizeboxclosebrace}%
\newsavebox{\hlfootnotesizeboxopenbrace}%
\newsavebox{\hlfootnotesizeboxbackslash}%
\newsavebox{\hlfootnotesizeboxlessthan}%
\newsavebox{\hlfootnotesizeboxgreaterthan}%
\newsavebox{\hlfootnotesizeboxdollar}%
\newsavebox{\hlfootnotesizeboxunderscore}%
\newsavebox{\hlfootnotesizeboxand}%
\newsavebox{\hlfootnotesizeboxhash}%
\newsavebox{\hlfootnotesizeboxat}%
\newsavebox{\hlfootnotesizeboxpercent}% 
\newsavebox{\hlfootnotesizeboxhat}%
\newsavebox{\hlfootnotesizeboxsinglequote}%
\newsavebox{\hlfootnotesizeboxbacktick}%

\setbox\hlfootnotesizeboxopenbrace=\hbox{\begin{footnotesize}\verb.{.\end{footnotesize}}%
\setbox\hlfootnotesizeboxclosebrace=\hbox{\begin{footnotesize}\verb.}.\end{footnotesize}}%
\setbox\hlfootnotesizeboxlessthan=\hbox{\begin{footnotesize}\verb.<.\end{footnotesize}}%
\setbox\hlfootnotesizeboxdollar=\hbox{\begin{footnotesize}\verb.$.\end{footnotesize}}%
\setbox\hlfootnotesizeboxunderscore=\hbox{\begin{footnotesize}\verb._.\end{footnotesize}}%
\setbox\hlfootnotesizeboxand=\hbox{\begin{footnotesize}\verb.&.\end{footnotesize}}%
\setbox\hlfootnotesizeboxhash=\hbox{\begin{footnotesize}\verb.#.\end{footnotesize}}%
\setbox\hlfootnotesizeboxat=\hbox{\begin{footnotesize}\verb.@.\end{footnotesize}}%
\setbox\hlfootnotesizeboxbackslash=\hbox{\begin{footnotesize}\verb.\.\end{footnotesize}}%
\setbox\hlfootnotesizeboxgreaterthan=\hbox{\begin{footnotesize}\verb.>.\end{footnotesize}}%
\setbox\hlfootnotesizeboxpercent=\hbox{\begin{footnotesize}\verb.%.\end{footnotesize}}%
\setbox\hlfootnotesizeboxhat=\hbox{\begin{footnotesize}\verb.^.\end{footnotesize}}%
\setbox\hlfootnotesizeboxsinglequote=\hbox{\begin{footnotesize}\verb.'.\end{footnotesize}}%
\setbox\hlfootnotesizeboxbacktick=\hbox{\begin{footnotesize}\verb.`.\end{footnotesize}}%
\setbox\hlfootnotesizeboxhat=\hbox{\begin{footnotesize}\verb.^.\end{footnotesize}}%



\newsavebox{\hlsmallboxclosebrace}%
\newsavebox{\hlsmallboxopenbrace}%
\newsavebox{\hlsmallboxbackslash}%
\newsavebox{\hlsmallboxlessthan}%
\newsavebox{\hlsmallboxgreaterthan}%
\newsavebox{\hlsmallboxdollar}%
\newsavebox{\hlsmallboxunderscore}%
\newsavebox{\hlsmallboxand}%
\newsavebox{\hlsmallboxhash}%
\newsavebox{\hlsmallboxat}%
\newsavebox{\hlsmallboxpercent}% 
\newsavebox{\hlsmallboxhat}%
\newsavebox{\hlsmallboxsinglequote}%
\newsavebox{\hlsmallboxbacktick}%

\setbox\hlsmallboxopenbrace=\hbox{\begin{small}\verb.{.\end{small}}%
\setbox\hlsmallboxclosebrace=\hbox{\begin{small}\verb.}.\end{small}}%
\setbox\hlsmallboxlessthan=\hbox{\begin{small}\verb.<.\end{small}}%
\setbox\hlsmallboxdollar=\hbox{\begin{small}\verb.$.\end{small}}%
\setbox\hlsmallboxunderscore=\hbox{\begin{small}\verb._.\end{small}}%
\setbox\hlsmallboxand=\hbox{\begin{small}\verb.&.\end{small}}%
\setbox\hlsmallboxhash=\hbox{\begin{small}\verb.#.\end{small}}%
\setbox\hlsmallboxat=\hbox{\begin{small}\verb.@.\end{small}}%
\setbox\hlsmallboxbackslash=\hbox{\begin{small}\verb.\.\end{small}}%
\setbox\hlsmallboxgreaterthan=\hbox{\begin{small}\verb.>.\end{small}}%
\setbox\hlsmallboxpercent=\hbox{\begin{small}\verb.%.\end{small}}%
\setbox\hlsmallboxhat=\hbox{\begin{small}\verb.^.\end{small}}%
\setbox\hlsmallboxsinglequote=\hbox{\begin{small}\verb.'.\end{small}}%
\setbox\hlsmallboxbacktick=\hbox{\begin{small}\verb.`.\end{small}}%
\setbox\hlsmallboxhat=\hbox{\begin{small}\verb.^.\end{small}}%



\newsavebox{\hllargeboxclosebrace}%
\newsavebox{\hllargeboxopenbrace}%
\newsavebox{\hllargeboxbackslash}%
\newsavebox{\hllargeboxlessthan}%
\newsavebox{\hllargeboxgreaterthan}%
\newsavebox{\hllargeboxdollar}%
\newsavebox{\hllargeboxunderscore}%
\newsavebox{\hllargeboxand}%
\newsavebox{\hllargeboxhash}%
\newsavebox{\hllargeboxat}%
\newsavebox{\hllargeboxpercent}% 
\newsavebox{\hllargeboxhat}%
\newsavebox{\hllargeboxsinglequote}%
\newsavebox{\hllargeboxbacktick}%

\setbox\hllargeboxopenbrace=\hbox{\begin{large}\verb.{.\end{large}}%
\setbox\hllargeboxclosebrace=\hbox{\begin{large}\verb.}.\end{large}}%
\setbox\hllargeboxlessthan=\hbox{\begin{large}\verb.<.\end{large}}%
\setbox\hllargeboxdollar=\hbox{\begin{large}\verb.$.\end{large}}%
\setbox\hllargeboxunderscore=\hbox{\begin{large}\verb._.\end{large}}%
\setbox\hllargeboxand=\hbox{\begin{large}\verb.&.\end{large}}%
\setbox\hllargeboxhash=\hbox{\begin{large}\verb.#.\end{large}}%
\setbox\hllargeboxat=\hbox{\begin{large}\verb.@.\end{large}}%
\setbox\hllargeboxbackslash=\hbox{\begin{large}\verb.\.\end{large}}%
\setbox\hllargeboxgreaterthan=\hbox{\begin{large}\verb.>.\end{large}}%
\setbox\hllargeboxpercent=\hbox{\begin{large}\verb.%.\end{large}}%
\setbox\hllargeboxhat=\hbox{\begin{large}\verb.^.\end{large}}%
\setbox\hllargeboxsinglequote=\hbox{\begin{large}\verb.'.\end{large}}%
\setbox\hllargeboxbacktick=\hbox{\begin{large}\verb.`.\end{large}}%
\setbox\hllargeboxhat=\hbox{\begin{large}\verb.^.\end{large}}%



\newsavebox{\hlLargeboxclosebrace}%
\newsavebox{\hlLargeboxopenbrace}%
\newsavebox{\hlLargeboxbackslash}%
\newsavebox{\hlLargeboxlessthan}%
\newsavebox{\hlLargeboxgreaterthan}%
\newsavebox{\hlLargeboxdollar}%
\newsavebox{\hlLargeboxunderscore}%
\newsavebox{\hlLargeboxand}%
\newsavebox{\hlLargeboxhash}%
\newsavebox{\hlLargeboxat}%
\newsavebox{\hlLargeboxpercent}% 
\newsavebox{\hlLargeboxhat}%
\newsavebox{\hlLargeboxsinglequote}%
\newsavebox{\hlLargeboxbacktick}%

\setbox\hlLargeboxopenbrace=\hbox{\begin{Large}\verb.{.\end{Large}}%
\setbox\hlLargeboxclosebrace=\hbox{\begin{Large}\verb.}.\end{Large}}%
\setbox\hlLargeboxlessthan=\hbox{\begin{Large}\verb.<.\end{Large}}%
\setbox\hlLargeboxdollar=\hbox{\begin{Large}\verb.$.\end{Large}}%
\setbox\hlLargeboxunderscore=\hbox{\begin{Large}\verb._.\end{Large}}%
\setbox\hlLargeboxand=\hbox{\begin{Large}\verb.&.\end{Large}}%
\setbox\hlLargeboxhash=\hbox{\begin{Large}\verb.#.\end{Large}}%
\setbox\hlLargeboxat=\hbox{\begin{Large}\verb.@.\end{Large}}%
\setbox\hlLargeboxbackslash=\hbox{\begin{Large}\verb.\.\end{Large}}%
\setbox\hlLargeboxgreaterthan=\hbox{\begin{Large}\verb.>.\end{Large}}%
\setbox\hlLargeboxpercent=\hbox{\begin{Large}\verb.%.\end{Large}}%
\setbox\hlLargeboxhat=\hbox{\begin{Large}\verb.^.\end{Large}}%
\setbox\hlLargeboxsinglequote=\hbox{\begin{Large}\verb.'.\end{Large}}%
\setbox\hlLargeboxbacktick=\hbox{\begin{Large}\verb.`.\end{Large}}%
\setbox\hlLargeboxhat=\hbox{\begin{Large}\verb.^.\end{Large}}%



\newsavebox{\hlLARGEboxclosebrace}%
\newsavebox{\hlLARGEboxopenbrace}%
\newsavebox{\hlLARGEboxbackslash}%
\newsavebox{\hlLARGEboxlessthan}%
\newsavebox{\hlLARGEboxgreaterthan}%
\newsavebox{\hlLARGEboxdollar}%
\newsavebox{\hlLARGEboxunderscore}%
\newsavebox{\hlLARGEboxand}%
\newsavebox{\hlLARGEboxhash}%
\newsavebox{\hlLARGEboxat}%
\newsavebox{\hlLARGEboxpercent}% 
\newsavebox{\hlLARGEboxhat}%
\newsavebox{\hlLARGEboxsinglequote}%
\newsavebox{\hlLARGEboxbacktick}%

\setbox\hlLARGEboxopenbrace=\hbox{\begin{LARGE}\verb.{.\end{LARGE}}%
\setbox\hlLARGEboxclosebrace=\hbox{\begin{LARGE}\verb.}.\end{LARGE}}%
\setbox\hlLARGEboxlessthan=\hbox{\begin{LARGE}\verb.<.\end{LARGE}}%
\setbox\hlLARGEboxdollar=\hbox{\begin{LARGE}\verb.$.\end{LARGE}}%
\setbox\hlLARGEboxunderscore=\hbox{\begin{LARGE}\verb._.\end{LARGE}}%
\setbox\hlLARGEboxand=\hbox{\begin{LARGE}\verb.&.\end{LARGE}}%
\setbox\hlLARGEboxhash=\hbox{\begin{LARGE}\verb.#.\end{LARGE}}%
\setbox\hlLARGEboxat=\hbox{\begin{LARGE}\verb.@.\end{LARGE}}%
\setbox\hlLARGEboxbackslash=\hbox{\begin{LARGE}\verb.\.\end{LARGE}}%
\setbox\hlLARGEboxgreaterthan=\hbox{\begin{LARGE}\verb.>.\end{LARGE}}%
\setbox\hlLARGEboxpercent=\hbox{\begin{LARGE}\verb.%.\end{LARGE}}%
\setbox\hlLARGEboxhat=\hbox{\begin{LARGE}\verb.^.\end{LARGE}}%
\setbox\hlLARGEboxsinglequote=\hbox{\begin{LARGE}\verb.'.\end{LARGE}}%
\setbox\hlLARGEboxbacktick=\hbox{\begin{LARGE}\verb.`.\end{LARGE}}%
\setbox\hlLARGEboxhat=\hbox{\begin{LARGE}\verb.^.\end{LARGE}}%



\newsavebox{\hlhugeboxclosebrace}%
\newsavebox{\hlhugeboxopenbrace}%
\newsavebox{\hlhugeboxbackslash}%
\newsavebox{\hlhugeboxlessthan}%
\newsavebox{\hlhugeboxgreaterthan}%
\newsavebox{\hlhugeboxdollar}%
\newsavebox{\hlhugeboxunderscore}%
\newsavebox{\hlhugeboxand}%
\newsavebox{\hlhugeboxhash}%
\newsavebox{\hlhugeboxat}%
\newsavebox{\hlhugeboxpercent}% 
\newsavebox{\hlhugeboxhat}%
\newsavebox{\hlhugeboxsinglequote}%
\newsavebox{\hlhugeboxbacktick}%

\setbox\hlhugeboxopenbrace=\hbox{\begin{huge}\verb.{.\end{huge}}%
\setbox\hlhugeboxclosebrace=\hbox{\begin{huge}\verb.}.\end{huge}}%
\setbox\hlhugeboxlessthan=\hbox{\begin{huge}\verb.<.\end{huge}}%
\setbox\hlhugeboxdollar=\hbox{\begin{huge}\verb.$.\end{huge}}%
\setbox\hlhugeboxunderscore=\hbox{\begin{huge}\verb._.\end{huge}}%
\setbox\hlhugeboxand=\hbox{\begin{huge}\verb.&.\end{huge}}%
\setbox\hlhugeboxhash=\hbox{\begin{huge}\verb.#.\end{huge}}%
\setbox\hlhugeboxat=\hbox{\begin{huge}\verb.@.\end{huge}}%
\setbox\hlhugeboxbackslash=\hbox{\begin{huge}\verb.\.\end{huge}}%
\setbox\hlhugeboxgreaterthan=\hbox{\begin{huge}\verb.>.\end{huge}}%
\setbox\hlhugeboxpercent=\hbox{\begin{huge}\verb.%.\end{huge}}%
\setbox\hlhugeboxhat=\hbox{\begin{huge}\verb.^.\end{huge}}%
\setbox\hlhugeboxsinglequote=\hbox{\begin{huge}\verb.'.\end{huge}}%
\setbox\hlhugeboxbacktick=\hbox{\begin{huge}\verb.`.\end{huge}}%
\setbox\hlhugeboxhat=\hbox{\begin{huge}\verb.^.\end{huge}}%



\newsavebox{\hlHugeboxclosebrace}%
\newsavebox{\hlHugeboxopenbrace}%
\newsavebox{\hlHugeboxbackslash}%
\newsavebox{\hlHugeboxlessthan}%
\newsavebox{\hlHugeboxgreaterthan}%
\newsavebox{\hlHugeboxdollar}%
\newsavebox{\hlHugeboxunderscore}%
\newsavebox{\hlHugeboxand}%
\newsavebox{\hlHugeboxhash}%
\newsavebox{\hlHugeboxat}%
\newsavebox{\hlHugeboxpercent}% 
\newsavebox{\hlHugeboxhat}%
\newsavebox{\hlHugeboxsinglequote}%
\newsavebox{\hlHugeboxbacktick}%

\setbox\hlHugeboxopenbrace=\hbox{\begin{Huge}\verb.{.\end{Huge}}%
\setbox\hlHugeboxclosebrace=\hbox{\begin{Huge}\verb.}.\end{Huge}}%
\setbox\hlHugeboxlessthan=\hbox{\begin{Huge}\verb.<.\end{Huge}}%
\setbox\hlHugeboxdollar=\hbox{\begin{Huge}\verb.$.\end{Huge}}%
\setbox\hlHugeboxunderscore=\hbox{\begin{Huge}\verb._.\end{Huge}}%
\setbox\hlHugeboxand=\hbox{\begin{Huge}\verb.&.\end{Huge}}%
\setbox\hlHugeboxhash=\hbox{\begin{Huge}\verb.#.\end{Huge}}%
\setbox\hlHugeboxat=\hbox{\begin{Huge}\verb.@.\end{Huge}}%
\setbox\hlHugeboxbackslash=\hbox{\begin{Huge}\verb.\.\end{Huge}}%
\setbox\hlHugeboxgreaterthan=\hbox{\begin{Huge}\verb.>.\end{Huge}}%
\setbox\hlHugeboxpercent=\hbox{\begin{Huge}\verb.%.\end{Huge}}%
\setbox\hlHugeboxhat=\hbox{\begin{Huge}\verb.^.\end{Huge}}%
\setbox\hlHugeboxsinglequote=\hbox{\begin{Huge}\verb.'.\end{Huge}}%
\setbox\hlHugeboxbacktick=\hbox{\begin{Huge}\verb.`.\end{Huge}}%
\setbox\hlHugeboxhat=\hbox{\begin{Huge}\verb.^.\end{Huge}}%
 

\def\urltilda{\kern -.15em\lower .7ex\hbox{\~{}}\kern .04em}%

\newcommand{\hlstd}[1]{\textcolor[rgb]{0,0,0}{#1}}%
\newcommand{\hlnum}[1]{\textcolor[rgb]{0.16,0.16,1}{#1}}
\newcommand{\hlesc}[1]{\textcolor[rgb]{1,0,1}{#1}}
\newcommand{\hlstr}[1]{\textcolor[rgb]{1,0,0}{#1}}
\newcommand{\hldstr}[1]{\textcolor[rgb]{0.51,0.51,0}{#1}}
\newcommand{\hlslc}[1]{\textcolor[rgb]{0.51,0.51,0.51}{\it{#1}}}
\newcommand{\hlcom}[1]{\textcolor[rgb]{0.51,0.51,0.51}{\it{#1}}}
\newcommand{\hldir}[1]{\textcolor[rgb]{0,0.51,0}{#1}}
\newcommand{\hlsym}[1]{\textcolor[rgb]{0,0,0}{#1}}
\newcommand{\hlline}[1]{\textcolor[rgb]{0.33,0.33,0.33}{#1}}
\newcommand{\hlkwa}[1]{\textcolor[rgb]{0,0,0}{\bf{#1}}}
\newcommand{\hlkwb}[1]{\textcolor[rgb]{0.51,0,0}{#1}}
\newcommand{\hlkwc}[1]{\textcolor[rgb]{0,0,0}{\bf{#1}}}
\newcommand{\hlkwd}[1]{\textcolor[rgb]{0,0,0.51}{#1}}

\newenvironment{Houtput}{\raggedright}{%
%
}
\usepackage[T1]{fontenc}
\usepackage[latin9]{inputenc}
\usepackage{color}
\definecolor{note_fontcolor}{rgb}{0.80078125, 0.80078125, 0.80078125}
\usepackage{graphicx}
\usepackage{setspace}
\usepackage[authoryear]{natbib}
\usepackage{nameref}

\makeatletter

%%%%%%%%%%%%%%%%%%%%%%%%%%%%%% LyX specific LaTeX commands.
%% The greyedout annotation environment
\newenvironment{lyxgreyedout}
  {\textcolor{note_fontcolor}\bgroup\ignorespaces}
  {\ignorespacesafterend\egroup}

%%%%%%%%%%%%%%%%%%%%%%%%%%%%%% Textclass specific LaTeX commands.
 %\usepackage{Sweave}
 \ifdefined\Sinput
 \else
 \IfFileExists{Sweave.sty}{
   \usepackage{Sweave}
 }{
   \usepackage{graphicx,fancyvrb}
   \DefineVerbatimEnvironment{Sinput}{Verbatim}{fontshape=sl}
   \DefineVerbatimEnvironment{Soutput}{Verbatim}{}
   \DefineVerbatimEnvironment{Scode}{Verbatim}{fontshape=sl}
   \newenvironment{Schunk}{}{}
   \newcommand{\Sconcordance}[1]{%
     \ifx\pdfoutput\undefined%
       \csname newcount\endcsname\pdfoutput\fi%
       \ifcase\pdfoutput\special{##1}%
     \else%
       \begingroup%
       \pdfcompresslevel=0%
       \immediate\pdfobj stream{##1}%
       \pdfcatalog{/SweaveConcordance \the\pdflastobj\space 0 R}%
       \endgroup%
     \fi}
 }
 \fi
 \usepackage{tikz}
\usetikzlibrary{external}
\tikzexternalize[mode=list and make]


%%%%%%%%%%%%%%%%%%%%%%%%%%%%%% User specified LaTeX commands.
\usepackage{ae}

%% maxwidth is the original width if it's less than linewidth
%% otherwise use linewidth (to make sure the graphics do not exceed the margin)
\def\maxwidth{%
\ifdim\Gin@nat@width>\linewidth
\linewidth
\else
\Gin@nat@width
\fi
}

\usepackage{todonotes}
\usepackage[thickspace]{SIunits}
\usepackage{movie15}

\usepackage{hyperref}
\AtBeginDocument{\renewcommand{\ref}[1]{\mbox{\autoref{#1}}}}

\@ifundefined{showcaptionsetup}{}{%
 \PassOptionsToPackage{caption=false}{subfig}}
\usepackage{subfig}
\makeatother

\usepackage{babel}
\begin{document}



\title{Cortical distance determines the perception of two competing types
of visual motion}

\maketitle
MOTION REVERAL


\author{Peter B. Meilstrup and Michael N. Shadlen}

Psysiology and Biophysics\\
University of Washington



%setup


%% \maxwidth has been defined in the preamble; see document settings
\setkeys{Gin}{width=\maxwidth}


\begin{abstract}
We did stuff.
\end{abstract}

\section{Introduction}




\section{General Methods}


\subsection{Observers}

Observers were one of the authors (P.B.M.) and SOME NUMBER OF na\"{i}ve
observers (). All had normal or corrected-to-normal vision.


\subsection{Equipment}

%data



%settings



Stimuli were presented on a flat CRT video monitor (ViewSonic PF790).
Its resolution was set to $800\times600\;\mathrm{pixels}$
over a display area of $35.5\times35.5\centi\meter$
and it used a $120\hertz$ refresh rate. Experiments
were programmed in MATLAB using the Psychtoolbox \citep{Brainard:1997gq}
and Eyelink toolbox extensions \citep{Cornelissen:2002wl}, along
with custom OpenGL code. Grayscale stimuli were shown using equal
red, green, and blue signals. The monitor was calibrated using a Tektronix
photometer. A 50\% gray background was chosen to lie at the midpoint
between mimimum ($0.1286 \candela\per\meter\squared$)
and maximum ($60.97 \candela\per\meter\squared$)
luminances, which were in turn measured as patches against the gray
background. A hardware lookup table with 10-bit resolution was constructed
to linearize the display luminance. 

Subjects sat behind a blackout curtain so that ambient illumination
was mostly due to the monitor and viewed the screen binocularly using
a chin and forehead rest with the eyes $60\centi\meter$
from the screen. Eye position was monitored using a video-based eye
tracker (EyeLink 1000; SR Research) using a sample rate of $1000\hertz$.
Eye movements were recorded but are not reported in this paper. Subjects
gave responses by turning a knob (PowerMate; Griffin Technologies)
with their preferred hand. 


\subsection{Stimuli\label{sub:stimuli}}

\begin{figure}
\subfloat[\label{fig:stimuli}Example stimuli in space-time form, where time
progresses down along the vertical axis. All plots depict a stimulus
containing a single element whose carrier motion is to the right.
Stimuli were \emph{congruent}, \emph{conterphase} or \emph{incongruent},
based on whether the direction of carrier agreed with that of long-range
motion. Counterphase stimuli are a superposition of congruent and
incongruent stimuli.]{\includegraphics{0_Users_peter_analysis_ConcentricDirectionQuest_writing_x_t_stimuli.pdf}}

\subfloat[\label{mov:stimuli}A demonstration of the three stimulus types, at
low density. In all cases the envelope motion is counterclockwise,
while the carrier motion varies.]{\includemovie[poster,repeat]{2in}{2in}{/Users/peter/analysis/ConcentricDirectionQuest/writing/demo_counter.mov}

}

\caption{Stimuli used in this report.}


\end{figure}


Example stimuli are shown in \nameref{mov:stimuli}. The stimuli consisted
of a number of identical elements arranged into a circle, each with
a direction of motion. A space-time diagram of the motion of a single
elment is shown in \nameref{fig:stimuli} ($x$ being taken along
the circle that the element travels on.) The motion of each elament
consisted of a set of 5
discrete appearances of an envelope, each offset by regular space
($\Delta x$) and time ($\Delta t$) intervals.  Within each envelope,
the carrier was continuously in motion with a constant temporal frequency
$\omega$. The temporal profile of each envelope was Gaussian, with
standard deviation $d/2$; at the peak of the temporal envelope the
carrier was in cosine phase. At right angles to the direction of motion,
the envelope is Gaussian envelope with standard deviation $w/2$.
Along the direction of motion, the carrier modulated by the spatial
envelope forms a Cauchy filter function \citep{Klein:1985rz} with
peak spatial frequency $f$. The equation describing the luminance
profile of a patch as a function of position and time is then: 
\[
L(x,y,t)=\mathrm{cos}^{n}(\mathrm{tan}^{-1}(fx/n))\mathrm{cos}(n\cdot\mathrm{tan}^{-1}(fx/n)+\omega t)e^{-(t/2d)^{2}-(y/2d)^{2}}
\]
 with the direction of motion along $x$. The spatial bandwidth parameter
$n$ was set to 4 for all stimuli. 

The examples in \nameref{mov:stimuli} have the following settings,
the same as used in \nameref{sec:reversal}: For all trials, $\Delta t = 100\milli\second$,
$\omega = 10 \hertz$,
and $d = 0.067\second$.
For stimuli at $10^{\circ}$ eccentricity, $f = 1.33 \mathrm{cyc}\per\degree$,
$\Delta x = 0.75 \degree$,
and $w = 0.75\degree$,
and these three parameters were scaled proportionately to eccentricity.

We used three types of motion stimuli, illustrated as ($x$,$t$)
diagrams in \nameref{fig:stimuli} and demonstrated in \ref{mov:stimuli}.
In \emph{congruent} stimuli the displacement of each successive presentation
of a motion element ($\Delta x$) agreed in direction with the short-range
motion contained in each element. In \emph{incongruent} stimuli, the
direction of displacement was opposite the direction of short-range
motion. In \emph{counterphase} stimuli, the motion elements contained
a counterphase flicker formed by superposing motion elements with
equal and opposite short-range motion content. That is, the counterphase
stimuli have the same spatial and temporal frequency distribution
as in congruent and incongruent stimuli, but their motion energy is
equivocal between opposite directions. The second stimulus in \ref{mov:stimuli}
shows counterphase local motion. The contrast of the local motion
elements was 70.7\% for congruent and incongruent trials, and 100\%
for trials using counterphase stimuli (that is, 50\% contrast in each
direction). These contrast values appeared to equalize the subjective
contrast of the stimuli of each type.


\section{Demonstrations and Subjective Observations\label{sec:lookatthis}}

\begin{lyxgreyedout}
Structurally I view this section as a sort of ``Experiment 0''.
To me it has the fewest sequencing problems I describe the stimulus
before this section, and if I describe the task after this section.
So task description would go into submethods for experiment 1. If
I have had some demonstrations before motivating Experiment 1, and
for the demonstrations to follow a description of how the stimuli
are constructed.

I think I want to talk about subjective observations in a model-neutral
matter here, whole perhaps coming back to model-relevant things suggested
by subjective appearance later in discussion (under the rubric of
``what kind of motion system is this stimulating?'')%
\end{lyxgreyedout}


It is worth commenting about some properties of this type of stimulus
that can be subjectively observed, and (so far as we can tell) aknowledged
by almost all colleagues we show our displays to, even though we do
not attempt to establish them quantitatively.

\nameref{mov:wheels} demonstrates the effect of eccentric viewing
on the subjective appearance of motion in the display. Two ``wheels''
are shown, each containing five moving elements. Each element has
a short range motion, and a long range displacement. In the wheel
on the right, the direction of short-range and long-range components
is in agreement (both counterclockwise) whereas for the wheel on the
left, the short range component is clockwise while the long range
component is counterclockwise. What is striking in this display is
that its appearance changes depending on the viewer's gaze. With the
eyes fixated on the point in the center of the left wheel, the elements
on both wheels appear to orbit counterclockwise. But if gaze is focused
on the right wheel, the left wheel now appears to rotate counterclockwise. 

\nameref{mov:singles} shows only one element in each circle, but
is otherwise identical to \nameref{mov:wheels}. In this case the
direction of gaze does does not cause a change in the perceived direction
of motion of the element on the left. Therefore it would appear that
it is not just eccentric viewing that causes short range motion to
dominate over long range, but also the \emph{density} of the elements
in the stimulus.

Taken together, these demonstrations motivate our investigation of
the interplay of three factors in determining motion perception: the
agreement of short-range and long-range movement, the eccentricity
of the stimulus, and the density of elements contained therein.

\begin{figure}
\subfloat[\label{mov:wheels}When fixating at the center of the left wheel,
both wheels appear to move in the same direction. But when fixating
the center of the right wheel, both wheels appear to move in opposite
directions. The appearance of the right wheel's movement reverses
depending on the viewing eccentricity.]{\includemovie[poster,repeat]{4in}{2	in}{/Users/peter/analysis/ConcentricDirectionQuest/writing/demo_counter.mov}

}

\subfloat[\label{mov:singles}Same stimulus as \nameref{mov:wheels} with only
one element used in each. The perceived direction of motion is independent
of eccentricity. ]{\includemovie[poster,repeat]{4in}{2in}{/Users/peter/analysis/ConcentricDirectionQuest/writing/demo_single.mov}}

\subfloat[\label{mov:eyemovements}As above, but there is no displacement of
the envelopes over time. In this case the sensation of motion appears
to be affected by fixational eye movements.]{\includemovie[poster,repeat]{4in}{2in}{/Users/peter/analysis/ConcentricDirectionQuest/writing/demo_eyemovements.mov}

}\caption{\label{fig:wheels}Demonstrations of subjective phenomena.}
\end{figure}


The primary phenomenon we will be exploring in this paper is the density-driven
reversal of the apparent direction of motion  that occurs in incongruent
stimuli (those whose short range and long range components are in
opposition.) When elements are widely spaced, the apparent direction
of motion follows that of the envelopes; as element spacing decreases,
the direction of motion changes to agree with that of the carrier.



We will proceed to quantify this reversal effect below. In order to
do so we ask subjects to make forced classifications of the stimulus'
overall direction (clockwise or counterclockwise.) However, the subjective
appearance of our stimulus can be complex, with several properties
that change aside from overall direction. Although we do not attempt
to quantify these details of the appearance of our stimuli, we feel
it is worth describing our impressions of them. Suitable individual
adjustments of element spacing and other properties appear to elicit
the same kinds of descriptions in everyone we have shown these stimuli
to.

When elements are closely spaced, it is generally impossible to see
the direction of envelope motion; the overall impression is of motion
the the direction of the carrier. However, the amount of subjective
flicker does appear to differ between incongruent and congruent motion,
with ingongruent motion having more flicker.

Increasing the inter-element spacing somewhat, we approach a critical
spacing, where for incongruent stimuli, subjects equivocate in their
reports of perveived motion direction, %
\begin{lyxgreyedout}
{[}NOTE: This would be the first introduction of ``critical distance.''
{]}%
\end{lyxgreyedout}
{} The subjective appearance of a critically spaced, incongruent motion
stimulus is somewhat variable. To some observers, it appears to first
move in the direction of the carrier, and then to reverse direction
shortly after stimulus onset. Although both the carrier and envelope
motions are (considered separately) consistent with an underlying
rigid rotation of a surface around the fixation point, or alternately
a rotation of the observer, the perception does not appear to obey
a rigid body constraint; elements can appear to move clockwise in
some parts of the circle and counterclockwise elsewhere. 

\begin{lyxgreyedout}
Further, when motion direction appears to reverse after stimulus onset,
the perceptual reversal (and its behavioral correlate) occurs much
faster than the perceptual change that occurs when separate moving
objects are reinterpreted as a smaller number of rigidly moving bodies
\citep{Anstis:2011vn}\emph{.} Nor is the perception of movement bistable;
a change in perceived direction for some element of the circle does
not, in general, cause the perceived motion of the entire stimulus
switch direction. Attention does play a role; for incongruent displays
with closely spaced elements, it is possible to direct attention to
one (or sometimes several) elements in the circle. These elements
are seen to move in the direction of envelope motion, but the rest
of the circle is seen to move in the contrary direction, making for
a somewhat paradoxical percept wherein movement is in opposite directions,
but individual elements are not seen to cross over or collide.%
\end{lyxgreyedout}


Increasing the spacings further to where element motion tends to dominate,
observers often describe the motion contained in the stimulus as belonging
to two components, one belonging to the motion of the elements themselves,
and the other being a kind of ``wind'' overlaid on top of the elements.
This seems similar to the perception elicited by a moving grating
superposed on a stationary pedestal \citep{Lu:2001fv}.

At even wider element spacings the envelope motion dominates the perception.
It becomes difficult to even tell the direction of motion of the carrier,
apart from the amount of flicker. That is, the ``wind'' seen at
slightly smaller spacings becomes less evident. This ``capture''
phemomenon was characterized by Hedges et al.  for single element
stimuli constructed similarly to ours. 

The subjective speed of the elements also changes with the element
spacing. As element spacing is reduced, the perceived speed of element
motion (as distinguished from the ``wind'') also seems to reduce,
until spacing near the critical distance, where the elements almost
appear to be at a near standstill over the brief duration of the stimulus.

Thus at very narrow and very wide element spacings it becomes difficult
to tell congruent motion apart from incongruent motion via the sensation
of motion direction. However, the sensation of subjective flicker
and smoothness of movement does tend to tell the two stimuli apart.
Incongruent stimuli have more flicker, at both narrow and wide element
spacings, and appear to move less smoothly. We have attempted to correlate
this sensation of smoothness or lack thereof with reports of the instantaneous
position of the elements,  without success.

\begin{lyxgreyedout}
Which demos are necessary to underscore these points?%
\end{lyxgreyedout}


\begin{lyxgreyedout}
This next paragraph doesn't really connect to anything unless I perform
an analysis of microsaccades in the task.... %
\end{lyxgreyedout}


As a final note, in some instances it appears that small fixational
eye movements have some influence over the appearance of these stimuli.
This is most easily demonstrated in a where the long range displacement
is zero but the carrier has a consistent direction, as shown in \nameref{mov:eyemovements}.
Viewing this stimulus with the elements in the periphery, one sees
a counterclockwise motion of the stimulus, which appears to fade and
slow over time. Hovever, large or small saccades seem to re-awaken
the movement, causing it to speed up again. The impression is similar
to that seen in versions of the peripheral drift illusion, such as
the ``rotating snakes'' illusion of Kitaoka \citep{Kitaoka:2007fv}.
However, while perceived motion in the peripheral drift illusion appears
to be driven by slight shifts of a stationary image on the retina,
the construction of our stimulus would tend to rule that out as a
mechanism; the motion is directly contained in the stimulus.

NOTE: Comparing further to rotating snakes would be overthinking,
but: each retinal slip has a direction, which (in the rotating snakes)
interacts with a chiral asymmetry of the stimulus to produce a coherent
direction of perceived motion. But our stimulus lacks this chiral
asymmetry; a retinal slip would have an opposite effect in each side
of the circle.

\pagebreak{}


\section{Experiment 1\label{sec:reversal}}

The demonstrations in\nameref{mov:wheels} and \nameref{mov:singles}
seem to suggest that the perceived reversal of motion direction is
a function not only of eccentricity viewing but of an interaction
between multiple moving elements. We explored this spatial interaction
by varying the distance between elements that were moving at a constant
eccentricity. 


\subsection{Methods}

%radii



Observers viewed stimuli at four different eccentricities (2.96, 4.44, 6.67, and 10.00$\degree$),
with parameters as described in \nameref{sub:stimuli}. Elements were
spaced evenly around a circle surrounding the fixation point; therefore
varying the spacing of the elements is equivalent to increasing the
number of elements in the stimulus. We ued the method of constant
stimuli. The stimulus in each trial with equal probability contained
either congruent, incongruent, or counterphase motion elements, with
the direction of long range motion being with equal probability clockwise
or counterclockwise. The set of element spacings was chosen to cover
the psychometric function for each observer, based on preliminary
sessions. Data shown is an aggregate of multiple sessions (at least
3 for each subject).


\subsubsection{Task}

\begin{figure}
\includegraphics{1_Users_peter_analysis_ConcentricDirectionQuest_writing_task.pdf}\caption{\label{fig:task}The timecourse of the task.}
\end{figure}


The timecourse of a trial is illustrated in \nameref{fig:task}. A
fixation point was presented. The computer then waited for the observer
to fixate. 250 ms after detecting fixation, the motion stimulus was
shown. After the motion stimulus concluded, observers indicated the
direction of perceived motion by turning a knob. If subjects blinked
or broke fixation before the offset of the motion stimulus, the trial
was aborted and reshuffled into the stimulus set to be repeated later
in the session. Observers were also asked to respond within a fixed
temporal window. Response latency was defined as the elapsed time
between motion onset and the knob being turned.  If the response latency
was more than $550\milli\second$
or less than $1050\milli\second$,
subjects received visual feedback that their response was either too
fast or too slow, and the trial was reshuffled into the stimulus set
to be repeated later in the session.

For congruent and counterphase stimuli, subjects received audio feedback
immediately after they gave a response, in the form of a high or low
tone depending on whether their responses agreed with the direction
of long-range motion. For incongruent stimuli, a neutral click was
given as feedback instead of a tone; observers were told that there
is no ``correct'' answer for those stimuli. This audio feedback
seemed to help subjects establish a rhythm through the experiment. 

Although we are most concerned with the appearance of incongruent
motion stimuli, we used an equal proportion of congruent, counterphase,
and incongruent stimuli in all experiments. We felt this was necessary
to verify that observers were judging the long range motion component
of the stimulus, rather than developing alternate strategies (such
as simply countermanding the perceived short range motion.) 



Subjects performed the task in sessions of at most 1 hour, divided
into 4 to 6 blocks of 10 to 200 trials each, and were prompted to
take a break between blocks. Subjects could also rest at any point
by simply delaying fixation. At the beginning of each block, the eye
tracking system was automatically recalibrated by asking the subject
to make saccades to a sequence of targets at randomly chosen locations
on the screen.


\subsection{Results}

\begin{figure}
%response-rates



%response-rate-plot


\tikzsetnextfilename{Rfigs/-response-rate-plot}

\tikzexternalfiledependsonfile{Rfigs/-response-rate-plot}{Rfigs/-response-rate-plot.tikz}
\input{Rfigs/-response-rate-plot.tikz}


\caption{\label{fig:rawdata}\label{fig:raw-response-rates}Percieved direction
of motion as a function of inter-element spacing. The vertical axis
plots the proportion of responses where the percieved direction of
motion agrees with the long-range component of the motion stimulus.
The horizontal axis the spacing between elements, in visual arc. Stimuli
plotted here were presented at 6.67 degrees eccentricity. Only responses
that fall within the time window are used in the computation of response
rates.}
\end{figure}
\begin{figure}
%response-time



%response-time-plot


\tikzsetnextfilename{Rfigs/-response-time-plot}

\tikzexternalfiledependsonfile{Rfigs/-response-time-plot}{Rfigs/-response-time-plot.tikz}
\input{Rfigs/-response-time-plot.tikz}


\caption{\label{fig:response-times}Responses to incongruent stimuli are color-coded
according to whether the responses agree with the direction of long-range
motion. All responses are shown, including those falling outside of
the response time window.}
\end{figure}



\subsubsection{Element spacing drives the perceptual reversal of perceived direction
for incongruent stimuli}

Here we display only the data from the $6.67\degree$ eccentricity.
In \nameref{fig:rawdata}we fold together clockwise and counterclockwise
motions and compute the proportion with which the overall motion was
perceived to be in teh same direction as the actual envelope motion.
This proportion is plotted as a function of inter-element spacing
for each of the three motion types (concruent, incongruent, and counterphase.)

The main effect evidenced is for incongruent motion, where the direction
of short-range motion is opposite that of the long-range motion. At
large inter-element spacings, observers judged the stimuli to move
according to its global direction of motion. However, as inter-element
spacing decreases,

For congruent motion, as expected, subjects are able to judge its
motion nearly perfectly. For counterphase motion, however, there is
a trend for increased errors at smaller spacings.  


\subsubsection{Critical spacing}

We fit cumulative logistic functions to subjects' responses, which
are overlaid on\nameref{fig:rawdata}. For robustness, we fixed the
lower asymptote, corresponding to responses in the direction of short-range
motion at 10\% and the upper asymptote, corresponding to responses
in the direction of long-range motion at 95\%.  This curve fitting
was implemented as generalized linear model, with binomial errors
and an inverse link function corresponding to a modified logistic
curve scaled to have the chosen asymptotes. We define the critical
spacing to be where the fitted curve crosses 50\%; this is indicated
in the graph with horizontal error bars. In all subjects this critical
spacing 


\subsubsection{Dependence on response time}

Some observers' responses varied with their response time. In \nameref{fig:response-times},
we show an aggregate of all data including data which fell outside
the response time window. Each trial is plotted as a single point,
color coded according to whether the observer judged the subjective
motion to be inagreement with the stimulus' short-range motion or
with its long-range motion (only data from incongruent motion trials
is shown, so these possibilities are mutually exclusive.) Points are
horizontally jittered to avoid overlap; we only tested discrete values
of element spacing. What is evident is that there is not only a dependence
on the target spacing, but for two subjects (D.A. and J.T.) there
is also a dependence on response time. For fast trials where obervers
responded sooner than about 400 ms from motion onset, most responses
agreed with the short-range component of the motion stimulus. But
for longer response times, observers tended to agree with the 

These observers' verbal descriptions of the stimulus tended to agree
with this temporal depencence. Observers reported that some stimuli
started with an apparent motion in one direction, but then the direction
appeared to reverse. Another frequent report was a sensation of second-guessing
oneself after giving an initial response.  

There may be a number of resons fot this dependence on response time.
It may be due to a longer time required for the long-range motion
system to activate  and may also simply be due to the fact that information
about short-range motion is present in the stimulus slightly earlier
than information about long-range motion (as the direction of long-range
motion is only established when the envelopes have been seen at two
locations,) although there was not a strong tendency for observers'
responses to counterphase motion trials to change as a function of
response time. Information about short and long-range motion may
be integrated in a desision process which favors long-range motion
as more information is gathered. Or other cues such as flicker may
be involved. However, there was not a tendency for observers to change
their reponse times as a function of element spacing, or other physical
properties of the stimulus.  For the purposes of investigating the
effect of element spacing, we set a 500 ms wide window beginning 400
ms from motion onset, for this and further experiments. Subjects received
feedback for whether their responses fell in this window




\subsection{Change in spacing with eccentricity\label{sec:eccentricity}}



%eccentricity


\begin{figure}
\caption{\label{fig:eccentricity-psychfuns}Response rates for incongruent
stimuli for each subject. Element spacing is plotted on the horizontal
axis, and response rate is plotted on the vertical axis. Colors indicate
stimuli tested at different eccentricities.}
\end{figure}


\begin{figure}
\caption{\label{fig:eccentricity-pse}Horizontal intercepts are computed and
shown with standard error bars.}
\end{figure}





\section{Experiment 3\label{sec:number}}

I'm working on this in a separate document at the moment.


\section{Changes in stimulus size}




\section{Discussion}

We should discuss the ways in which this stimulus differs from crowding
(and yet, how the phenomena are similar.) Draw the connection between
``feature spread'' and crowding. I.e. Several studies have 

First there is the question of what kind of motion system the long-range
component is engaging. We have tried to avoid the tempting terms ``local''
and ``global'' to describe the two 

\begin{singlespace}
Another form of large-scale integration of short range motion occurs
in the computation of optic flow, which term  describes a pattern
of motion over the entire visual field that is consistent with being
generated by the motion of an observer through the environment, opposed
to the motion of objects relative to a stationary observer. The simplest
examples of optic flow stimuli are patterns such as global expantion
or contraction, or rotation around a central point. The patterns of
motion that in our motion illusion are consistent with a rotation
around the axis parallel to the viewer's gaze, so an optic flow mechanism
is likely being driven my our stimulus. This raises the question of
whether optic flow mechanisms are involved in the illusion of reversal.
{[}some sense of{]} However, {[}no second order global motion{]} {[}our
experiment on number versus shape{]} {[}however we found that...{]}

In the monkey, it appears that {[}the dorsal subdivision of{]} area
MST is specialized for optic flow patterns. 
\end{singlespace}


\subsection{Classification of this stimulus among the various types of motion
stimuli}

Some models for detection of non-luminance-defined motion posit a
mechanism that operates much like first-order, short-range motion,
but applies some (space-time separable) nonlinearity to the image
before applying the motion energy analysis. This would be called ``second
order'' motion system in the scheme of \citet{Lu:1995la}. We noted
in\nameref{sec:lookatthis}that for a lone element, it is more difficult
to discern the direciton of its short-range motion than the clear
long-range movement. By adjusting the parameters to further favor
the long-range component, we can produce stimuli where the direction
of the short-range motion is nearly undetectable, yet these stimuli
still elicit a motion after-effect according only to the direction
of hort range motion. Nonetheless, these stimuli still elicit a direction-elective
response in area MT only to the short-range component\citep{Shadlen:1993ne}.
motion and independent the direction of long-range motion. Thus, while
the have added  have quantitatively demonstrated the presence of
this phenomenon, which they term 'motion capture.' That is, the long-range
motion is capable of masking and obscuring short range motion. Additionally,
a mechanism driven by a single input nonlinearity (and in some formulations,
any purely local mechanism \citep{Zanker:1993uq} would be driven
by first-order motion stimuli as well as second-order;  this is consistent
with results showing that adding second-order noise, or even coherent
second-order motion, does not interfere with the decection of first-order
motion, but first-order noise does interfere with detection of second-order
motion \citep{Edwards:1995fk,Cassanello:2011uq}. However, this behavior
is not consistent with our separate long-range and short-range motion;
local motion opposes it. So while a second-order mechanism seems to
exist and is able to detect some forms of non-Fourier motion, we do
not think a second-order mechanism plays a significant explanatory
role in our illusion and is not responsible for the illusory reversal.





\citet{Bex:2003ix}investigated crowding for discrimination of moving
objects in a variation of the typical crowding task where targets
and flankers moved around an annulus. They found that the size of
the region of interference between the target and a lateral flanker
was invariant with target speed (if anything, there was a maximum
at 2-4 ``angular degrees per frame'' (at 75 Hz; at 8 degrees ecentricity,
2 degrees per frame \textasciitilde{}21 degrees-of-visual-angle per
second. That's a lot faster than out stimuli.) They did find that
the zone of interference around a moving target was asymmetric contingent
on motion; a flanker moving ahead of the target crowded at a greater
distance than a flanker that trailed the target. If this were simply
due to temporal summation (i.e. motion blurring,) ``we would expect
crowding to increase with speed because motion blur increases with
speed of sharp objects'' 



\citet{Bex:2005zm}show that crowding applies to the identificaiton
of direciton of motion of a textured patch within a window. Since
the stimulus in that study does not include a global direction component,
it is not (all of our local motion features are identical.) Actually
the format of this study shares many similarities to our own, but
what they establish is the spatial interference of flanking (short-range)
motion on identification of neighboring patches of (short--range)
motion, wherewas what be identify in this study is the interference
of flanking (short-range) motion in the identification of (long-range)
motion -- this is an interesting distinction, as it shows that spatial
interference happens in a way that prevents 















\bibliographystyle{plain}
\bibliography{xampl,2_Users_peter_analysis_ConcentricDirectionQuest_writing_bibliography}

\end{document}
