\batchmode
\documentclass[english,noae]{article}
\newcommand{\hlnumber}[1]{\textcolor[rgb]{0.0823529411764706,0.0784313725490196,0.709803921568627}{#1}}%
\newcommand{\hlfunctioncall}[1]{\textcolor[rgb]{1,0,0}{#1}}%
\newcommand{\hlstring}[1]{\textcolor[rgb]{0.6,0.6,1}{#1}}%
\newcommand{\hlkeyword}[1]{\textcolor[rgb]{0,0,0}{\textbf{#1}}}%
\newcommand{\hlargument}[1]{\textcolor[rgb]{0.694117647058824,0.247058823529412,0.0196078431372549}{#1}}%
\newcommand{\hlcomment}[1]{\textcolor[rgb]{0.8,0.8,0.8}{#1}}%
\newcommand{\hlroxygencomment}[1]{\textcolor[rgb]{0,0.592156862745098,1}{#1}}%
\newcommand{\hlformalargs}[1]{\textcolor[rgb]{0.0705882352941176,0.713725490196078,0.0705882352941176}{#1}}%
\newcommand{\hleqformalargs}[1]{\textcolor[rgb]{0.0705882352941176,0.713725490196078,0.0705882352941176}{#1}}%
\newcommand{\hlassignement}[1]{\textcolor[rgb]{0.215686274509804,0.215686274509804,0.384313725490196}{\textbf{#1}}}%
\newcommand{\hlpackage}[1]{\textcolor[rgb]{0.588235294117647,0.713725490196078,0.145098039215686}{#1}}%
\newcommand{\hlslot}[1]{\textit{#1}}%
\newcommand{\hlsymbol}[1]{\textcolor[rgb]{0,0,0}{#1}}%
\newcommand{\hlprompt}[1]{\textcolor[rgb]{0,0,0}{#1}}%

\usepackage{color}%
 
\newsavebox{\hlnormalsizeboxclosebrace}%
\newsavebox{\hlnormalsizeboxopenbrace}%
\newsavebox{\hlnormalsizeboxbackslash}%
\newsavebox{\hlnormalsizeboxlessthan}%
\newsavebox{\hlnormalsizeboxgreaterthan}%
\newsavebox{\hlnormalsizeboxdollar}%
\newsavebox{\hlnormalsizeboxunderscore}%
\newsavebox{\hlnormalsizeboxand}%
\newsavebox{\hlnormalsizeboxhash}%
\newsavebox{\hlnormalsizeboxat}%
\newsavebox{\hlnormalsizeboxpercent}% 
\newsavebox{\hlnormalsizeboxhat}%
\newsavebox{\hlnormalsizeboxsinglequote}%
\newsavebox{\hlnormalsizeboxbacktick}%

\setbox\hlnormalsizeboxopenbrace=\hbox{\begin{normalsize}\verb.{.\end{normalsize}}%
\setbox\hlnormalsizeboxclosebrace=\hbox{\begin{normalsize}\verb.}.\end{normalsize}}%
\setbox\hlnormalsizeboxlessthan=\hbox{\begin{normalsize}\verb.<.\end{normalsize}}%
\setbox\hlnormalsizeboxdollar=\hbox{\begin{normalsize}\verb.$.\end{normalsize}}%
\setbox\hlnormalsizeboxunderscore=\hbox{\begin{normalsize}\verb._.\end{normalsize}}%
\setbox\hlnormalsizeboxand=\hbox{\begin{normalsize}\verb.&.\end{normalsize}}%
\setbox\hlnormalsizeboxhash=\hbox{\begin{normalsize}\verb.#.\end{normalsize}}%
\setbox\hlnormalsizeboxat=\hbox{\begin{normalsize}\verb.@.\end{normalsize}}%
\setbox\hlnormalsizeboxbackslash=\hbox{\begin{normalsize}\verb.\.\end{normalsize}}%
\setbox\hlnormalsizeboxgreaterthan=\hbox{\begin{normalsize}\verb.>.\end{normalsize}}%
\setbox\hlnormalsizeboxpercent=\hbox{\begin{normalsize}\verb.%.\end{normalsize}}%
\setbox\hlnormalsizeboxhat=\hbox{\begin{normalsize}\verb.^.\end{normalsize}}%
\setbox\hlnormalsizeboxsinglequote=\hbox{\begin{normalsize}\verb.'.\end{normalsize}}%
\setbox\hlnormalsizeboxbacktick=\hbox{\begin{normalsize}\verb.`.\end{normalsize}}%
\setbox\hlnormalsizeboxhat=\hbox{\begin{normalsize}\verb.^.\end{normalsize}}%



\newsavebox{\hltinyboxclosebrace}%
\newsavebox{\hltinyboxopenbrace}%
\newsavebox{\hltinyboxbackslash}%
\newsavebox{\hltinyboxlessthan}%
\newsavebox{\hltinyboxgreaterthan}%
\newsavebox{\hltinyboxdollar}%
\newsavebox{\hltinyboxunderscore}%
\newsavebox{\hltinyboxand}%
\newsavebox{\hltinyboxhash}%
\newsavebox{\hltinyboxat}%
\newsavebox{\hltinyboxpercent}% 
\newsavebox{\hltinyboxhat}%
\newsavebox{\hltinyboxsinglequote}%
\newsavebox{\hltinyboxbacktick}%

\setbox\hltinyboxopenbrace=\hbox{\begin{tiny}\verb.{.\end{tiny}}%
\setbox\hltinyboxclosebrace=\hbox{\begin{tiny}\verb.}.\end{tiny}}%
\setbox\hltinyboxlessthan=\hbox{\begin{tiny}\verb.<.\end{tiny}}%
\setbox\hltinyboxdollar=\hbox{\begin{tiny}\verb.$.\end{tiny}}%
\setbox\hltinyboxunderscore=\hbox{\begin{tiny}\verb._.\end{tiny}}%
\setbox\hltinyboxand=\hbox{\begin{tiny}\verb.&.\end{tiny}}%
\setbox\hltinyboxhash=\hbox{\begin{tiny}\verb.#.\end{tiny}}%
\setbox\hltinyboxat=\hbox{\begin{tiny}\verb.@.\end{tiny}}%
\setbox\hltinyboxbackslash=\hbox{\begin{tiny}\verb.\.\end{tiny}}%
\setbox\hltinyboxgreaterthan=\hbox{\begin{tiny}\verb.>.\end{tiny}}%
\setbox\hltinyboxpercent=\hbox{\begin{tiny}\verb.%.\end{tiny}}%
\setbox\hltinyboxhat=\hbox{\begin{tiny}\verb.^.\end{tiny}}%
\setbox\hltinyboxsinglequote=\hbox{\begin{tiny}\verb.'.\end{tiny}}%
\setbox\hltinyboxbacktick=\hbox{\begin{tiny}\verb.`.\end{tiny}}%
\setbox\hltinyboxhat=\hbox{\begin{tiny}\verb.^.\end{tiny}}%



\newsavebox{\hlscriptsizeboxclosebrace}%
\newsavebox{\hlscriptsizeboxopenbrace}%
\newsavebox{\hlscriptsizeboxbackslash}%
\newsavebox{\hlscriptsizeboxlessthan}%
\newsavebox{\hlscriptsizeboxgreaterthan}%
\newsavebox{\hlscriptsizeboxdollar}%
\newsavebox{\hlscriptsizeboxunderscore}%
\newsavebox{\hlscriptsizeboxand}%
\newsavebox{\hlscriptsizeboxhash}%
\newsavebox{\hlscriptsizeboxat}%
\newsavebox{\hlscriptsizeboxpercent}% 
\newsavebox{\hlscriptsizeboxhat}%
\newsavebox{\hlscriptsizeboxsinglequote}%
\newsavebox{\hlscriptsizeboxbacktick}%

\setbox\hlscriptsizeboxopenbrace=\hbox{\begin{scriptsize}\verb.{.\end{scriptsize}}%
\setbox\hlscriptsizeboxclosebrace=\hbox{\begin{scriptsize}\verb.}.\end{scriptsize}}%
\setbox\hlscriptsizeboxlessthan=\hbox{\begin{scriptsize}\verb.<.\end{scriptsize}}%
\setbox\hlscriptsizeboxdollar=\hbox{\begin{scriptsize}\verb.$.\end{scriptsize}}%
\setbox\hlscriptsizeboxunderscore=\hbox{\begin{scriptsize}\verb._.\end{scriptsize}}%
\setbox\hlscriptsizeboxand=\hbox{\begin{scriptsize}\verb.&.\end{scriptsize}}%
\setbox\hlscriptsizeboxhash=\hbox{\begin{scriptsize}\verb.#.\end{scriptsize}}%
\setbox\hlscriptsizeboxat=\hbox{\begin{scriptsize}\verb.@.\end{scriptsize}}%
\setbox\hlscriptsizeboxbackslash=\hbox{\begin{scriptsize}\verb.\.\end{scriptsize}}%
\setbox\hlscriptsizeboxgreaterthan=\hbox{\begin{scriptsize}\verb.>.\end{scriptsize}}%
\setbox\hlscriptsizeboxpercent=\hbox{\begin{scriptsize}\verb.%.\end{scriptsize}}%
\setbox\hlscriptsizeboxhat=\hbox{\begin{scriptsize}\verb.^.\end{scriptsize}}%
\setbox\hlscriptsizeboxsinglequote=\hbox{\begin{scriptsize}\verb.'.\end{scriptsize}}%
\setbox\hlscriptsizeboxbacktick=\hbox{\begin{scriptsize}\verb.`.\end{scriptsize}}%
\setbox\hlscriptsizeboxhat=\hbox{\begin{scriptsize}\verb.^.\end{scriptsize}}%



\newsavebox{\hlfootnotesizeboxclosebrace}%
\newsavebox{\hlfootnotesizeboxopenbrace}%
\newsavebox{\hlfootnotesizeboxbackslash}%
\newsavebox{\hlfootnotesizeboxlessthan}%
\newsavebox{\hlfootnotesizeboxgreaterthan}%
\newsavebox{\hlfootnotesizeboxdollar}%
\newsavebox{\hlfootnotesizeboxunderscore}%
\newsavebox{\hlfootnotesizeboxand}%
\newsavebox{\hlfootnotesizeboxhash}%
\newsavebox{\hlfootnotesizeboxat}%
\newsavebox{\hlfootnotesizeboxpercent}% 
\newsavebox{\hlfootnotesizeboxhat}%
\newsavebox{\hlfootnotesizeboxsinglequote}%
\newsavebox{\hlfootnotesizeboxbacktick}%

\setbox\hlfootnotesizeboxopenbrace=\hbox{\begin{footnotesize}\verb.{.\end{footnotesize}}%
\setbox\hlfootnotesizeboxclosebrace=\hbox{\begin{footnotesize}\verb.}.\end{footnotesize}}%
\setbox\hlfootnotesizeboxlessthan=\hbox{\begin{footnotesize}\verb.<.\end{footnotesize}}%
\setbox\hlfootnotesizeboxdollar=\hbox{\begin{footnotesize}\verb.$.\end{footnotesize}}%
\setbox\hlfootnotesizeboxunderscore=\hbox{\begin{footnotesize}\verb._.\end{footnotesize}}%
\setbox\hlfootnotesizeboxand=\hbox{\begin{footnotesize}\verb.&.\end{footnotesize}}%
\setbox\hlfootnotesizeboxhash=\hbox{\begin{footnotesize}\verb.#.\end{footnotesize}}%
\setbox\hlfootnotesizeboxat=\hbox{\begin{footnotesize}\verb.@.\end{footnotesize}}%
\setbox\hlfootnotesizeboxbackslash=\hbox{\begin{footnotesize}\verb.\.\end{footnotesize}}%
\setbox\hlfootnotesizeboxgreaterthan=\hbox{\begin{footnotesize}\verb.>.\end{footnotesize}}%
\setbox\hlfootnotesizeboxpercent=\hbox{\begin{footnotesize}\verb.%.\end{footnotesize}}%
\setbox\hlfootnotesizeboxhat=\hbox{\begin{footnotesize}\verb.^.\end{footnotesize}}%
\setbox\hlfootnotesizeboxsinglequote=\hbox{\begin{footnotesize}\verb.'.\end{footnotesize}}%
\setbox\hlfootnotesizeboxbacktick=\hbox{\begin{footnotesize}\verb.`.\end{footnotesize}}%
\setbox\hlfootnotesizeboxhat=\hbox{\begin{footnotesize}\verb.^.\end{footnotesize}}%



\newsavebox{\hlsmallboxclosebrace}%
\newsavebox{\hlsmallboxopenbrace}%
\newsavebox{\hlsmallboxbackslash}%
\newsavebox{\hlsmallboxlessthan}%
\newsavebox{\hlsmallboxgreaterthan}%
\newsavebox{\hlsmallboxdollar}%
\newsavebox{\hlsmallboxunderscore}%
\newsavebox{\hlsmallboxand}%
\newsavebox{\hlsmallboxhash}%
\newsavebox{\hlsmallboxat}%
\newsavebox{\hlsmallboxpercent}% 
\newsavebox{\hlsmallboxhat}%
\newsavebox{\hlsmallboxsinglequote}%
\newsavebox{\hlsmallboxbacktick}%

\setbox\hlsmallboxopenbrace=\hbox{\begin{small}\verb.{.\end{small}}%
\setbox\hlsmallboxclosebrace=\hbox{\begin{small}\verb.}.\end{small}}%
\setbox\hlsmallboxlessthan=\hbox{\begin{small}\verb.<.\end{small}}%
\setbox\hlsmallboxdollar=\hbox{\begin{small}\verb.$.\end{small}}%
\setbox\hlsmallboxunderscore=\hbox{\begin{small}\verb._.\end{small}}%
\setbox\hlsmallboxand=\hbox{\begin{small}\verb.&.\end{small}}%
\setbox\hlsmallboxhash=\hbox{\begin{small}\verb.#.\end{small}}%
\setbox\hlsmallboxat=\hbox{\begin{small}\verb.@.\end{small}}%
\setbox\hlsmallboxbackslash=\hbox{\begin{small}\verb.\.\end{small}}%
\setbox\hlsmallboxgreaterthan=\hbox{\begin{small}\verb.>.\end{small}}%
\setbox\hlsmallboxpercent=\hbox{\begin{small}\verb.%.\end{small}}%
\setbox\hlsmallboxhat=\hbox{\begin{small}\verb.^.\end{small}}%
\setbox\hlsmallboxsinglequote=\hbox{\begin{small}\verb.'.\end{small}}%
\setbox\hlsmallboxbacktick=\hbox{\begin{small}\verb.`.\end{small}}%
\setbox\hlsmallboxhat=\hbox{\begin{small}\verb.^.\end{small}}%



\newsavebox{\hllargeboxclosebrace}%
\newsavebox{\hllargeboxopenbrace}%
\newsavebox{\hllargeboxbackslash}%
\newsavebox{\hllargeboxlessthan}%
\newsavebox{\hllargeboxgreaterthan}%
\newsavebox{\hllargeboxdollar}%
\newsavebox{\hllargeboxunderscore}%
\newsavebox{\hllargeboxand}%
\newsavebox{\hllargeboxhash}%
\newsavebox{\hllargeboxat}%
\newsavebox{\hllargeboxpercent}% 
\newsavebox{\hllargeboxhat}%
\newsavebox{\hllargeboxsinglequote}%
\newsavebox{\hllargeboxbacktick}%

\setbox\hllargeboxopenbrace=\hbox{\begin{large}\verb.{.\end{large}}%
\setbox\hllargeboxclosebrace=\hbox{\begin{large}\verb.}.\end{large}}%
\setbox\hllargeboxlessthan=\hbox{\begin{large}\verb.<.\end{large}}%
\setbox\hllargeboxdollar=\hbox{\begin{large}\verb.$.\end{large}}%
\setbox\hllargeboxunderscore=\hbox{\begin{large}\verb._.\end{large}}%
\setbox\hllargeboxand=\hbox{\begin{large}\verb.&.\end{large}}%
\setbox\hllargeboxhash=\hbox{\begin{large}\verb.#.\end{large}}%
\setbox\hllargeboxat=\hbox{\begin{large}\verb.@.\end{large}}%
\setbox\hllargeboxbackslash=\hbox{\begin{large}\verb.\.\end{large}}%
\setbox\hllargeboxgreaterthan=\hbox{\begin{large}\verb.>.\end{large}}%
\setbox\hllargeboxpercent=\hbox{\begin{large}\verb.%.\end{large}}%
\setbox\hllargeboxhat=\hbox{\begin{large}\verb.^.\end{large}}%
\setbox\hllargeboxsinglequote=\hbox{\begin{large}\verb.'.\end{large}}%
\setbox\hllargeboxbacktick=\hbox{\begin{large}\verb.`.\end{large}}%
\setbox\hllargeboxhat=\hbox{\begin{large}\verb.^.\end{large}}%



\newsavebox{\hlLargeboxclosebrace}%
\newsavebox{\hlLargeboxopenbrace}%
\newsavebox{\hlLargeboxbackslash}%
\newsavebox{\hlLargeboxlessthan}%
\newsavebox{\hlLargeboxgreaterthan}%
\newsavebox{\hlLargeboxdollar}%
\newsavebox{\hlLargeboxunderscore}%
\newsavebox{\hlLargeboxand}%
\newsavebox{\hlLargeboxhash}%
\newsavebox{\hlLargeboxat}%
\newsavebox{\hlLargeboxpercent}% 
\newsavebox{\hlLargeboxhat}%
\newsavebox{\hlLargeboxsinglequote}%
\newsavebox{\hlLargeboxbacktick}%

\setbox\hlLargeboxopenbrace=\hbox{\begin{Large}\verb.{.\end{Large}}%
\setbox\hlLargeboxclosebrace=\hbox{\begin{Large}\verb.}.\end{Large}}%
\setbox\hlLargeboxlessthan=\hbox{\begin{Large}\verb.<.\end{Large}}%
\setbox\hlLargeboxdollar=\hbox{\begin{Large}\verb.$.\end{Large}}%
\setbox\hlLargeboxunderscore=\hbox{\begin{Large}\verb._.\end{Large}}%
\setbox\hlLargeboxand=\hbox{\begin{Large}\verb.&.\end{Large}}%
\setbox\hlLargeboxhash=\hbox{\begin{Large}\verb.#.\end{Large}}%
\setbox\hlLargeboxat=\hbox{\begin{Large}\verb.@.\end{Large}}%
\setbox\hlLargeboxbackslash=\hbox{\begin{Large}\verb.\.\end{Large}}%
\setbox\hlLargeboxgreaterthan=\hbox{\begin{Large}\verb.>.\end{Large}}%
\setbox\hlLargeboxpercent=\hbox{\begin{Large}\verb.%.\end{Large}}%
\setbox\hlLargeboxhat=\hbox{\begin{Large}\verb.^.\end{Large}}%
\setbox\hlLargeboxsinglequote=\hbox{\begin{Large}\verb.'.\end{Large}}%
\setbox\hlLargeboxbacktick=\hbox{\begin{Large}\verb.`.\end{Large}}%
\setbox\hlLargeboxhat=\hbox{\begin{Large}\verb.^.\end{Large}}%



\newsavebox{\hlLARGEboxclosebrace}%
\newsavebox{\hlLARGEboxopenbrace}%
\newsavebox{\hlLARGEboxbackslash}%
\newsavebox{\hlLARGEboxlessthan}%
\newsavebox{\hlLARGEboxgreaterthan}%
\newsavebox{\hlLARGEboxdollar}%
\newsavebox{\hlLARGEboxunderscore}%
\newsavebox{\hlLARGEboxand}%
\newsavebox{\hlLARGEboxhash}%
\newsavebox{\hlLARGEboxat}%
\newsavebox{\hlLARGEboxpercent}% 
\newsavebox{\hlLARGEboxhat}%
\newsavebox{\hlLARGEboxsinglequote}%
\newsavebox{\hlLARGEboxbacktick}%

\setbox\hlLARGEboxopenbrace=\hbox{\begin{LARGE}\verb.{.\end{LARGE}}%
\setbox\hlLARGEboxclosebrace=\hbox{\begin{LARGE}\verb.}.\end{LARGE}}%
\setbox\hlLARGEboxlessthan=\hbox{\begin{LARGE}\verb.<.\end{LARGE}}%
\setbox\hlLARGEboxdollar=\hbox{\begin{LARGE}\verb.$.\end{LARGE}}%
\setbox\hlLARGEboxunderscore=\hbox{\begin{LARGE}\verb._.\end{LARGE}}%
\setbox\hlLARGEboxand=\hbox{\begin{LARGE}\verb.&.\end{LARGE}}%
\setbox\hlLARGEboxhash=\hbox{\begin{LARGE}\verb.#.\end{LARGE}}%
\setbox\hlLARGEboxat=\hbox{\begin{LARGE}\verb.@.\end{LARGE}}%
\setbox\hlLARGEboxbackslash=\hbox{\begin{LARGE}\verb.\.\end{LARGE}}%
\setbox\hlLARGEboxgreaterthan=\hbox{\begin{LARGE}\verb.>.\end{LARGE}}%
\setbox\hlLARGEboxpercent=\hbox{\begin{LARGE}\verb.%.\end{LARGE}}%
\setbox\hlLARGEboxhat=\hbox{\begin{LARGE}\verb.^.\end{LARGE}}%
\setbox\hlLARGEboxsinglequote=\hbox{\begin{LARGE}\verb.'.\end{LARGE}}%
\setbox\hlLARGEboxbacktick=\hbox{\begin{LARGE}\verb.`.\end{LARGE}}%
\setbox\hlLARGEboxhat=\hbox{\begin{LARGE}\verb.^.\end{LARGE}}%



\newsavebox{\hlhugeboxclosebrace}%
\newsavebox{\hlhugeboxopenbrace}%
\newsavebox{\hlhugeboxbackslash}%
\newsavebox{\hlhugeboxlessthan}%
\newsavebox{\hlhugeboxgreaterthan}%
\newsavebox{\hlhugeboxdollar}%
\newsavebox{\hlhugeboxunderscore}%
\newsavebox{\hlhugeboxand}%
\newsavebox{\hlhugeboxhash}%
\newsavebox{\hlhugeboxat}%
\newsavebox{\hlhugeboxpercent}% 
\newsavebox{\hlhugeboxhat}%
\newsavebox{\hlhugeboxsinglequote}%
\newsavebox{\hlhugeboxbacktick}%

\setbox\hlhugeboxopenbrace=\hbox{\begin{huge}\verb.{.\end{huge}}%
\setbox\hlhugeboxclosebrace=\hbox{\begin{huge}\verb.}.\end{huge}}%
\setbox\hlhugeboxlessthan=\hbox{\begin{huge}\verb.<.\end{huge}}%
\setbox\hlhugeboxdollar=\hbox{\begin{huge}\verb.$.\end{huge}}%
\setbox\hlhugeboxunderscore=\hbox{\begin{huge}\verb._.\end{huge}}%
\setbox\hlhugeboxand=\hbox{\begin{huge}\verb.&.\end{huge}}%
\setbox\hlhugeboxhash=\hbox{\begin{huge}\verb.#.\end{huge}}%
\setbox\hlhugeboxat=\hbox{\begin{huge}\verb.@.\end{huge}}%
\setbox\hlhugeboxbackslash=\hbox{\begin{huge}\verb.\.\end{huge}}%
\setbox\hlhugeboxgreaterthan=\hbox{\begin{huge}\verb.>.\end{huge}}%
\setbox\hlhugeboxpercent=\hbox{\begin{huge}\verb.%.\end{huge}}%
\setbox\hlhugeboxhat=\hbox{\begin{huge}\verb.^.\end{huge}}%
\setbox\hlhugeboxsinglequote=\hbox{\begin{huge}\verb.'.\end{huge}}%
\setbox\hlhugeboxbacktick=\hbox{\begin{huge}\verb.`.\end{huge}}%
\setbox\hlhugeboxhat=\hbox{\begin{huge}\verb.^.\end{huge}}%



\newsavebox{\hlHugeboxclosebrace}%
\newsavebox{\hlHugeboxopenbrace}%
\newsavebox{\hlHugeboxbackslash}%
\newsavebox{\hlHugeboxlessthan}%
\newsavebox{\hlHugeboxgreaterthan}%
\newsavebox{\hlHugeboxdollar}%
\newsavebox{\hlHugeboxunderscore}%
\newsavebox{\hlHugeboxand}%
\newsavebox{\hlHugeboxhash}%
\newsavebox{\hlHugeboxat}%
\newsavebox{\hlHugeboxpercent}% 
\newsavebox{\hlHugeboxhat}%
\newsavebox{\hlHugeboxsinglequote}%
\newsavebox{\hlHugeboxbacktick}%

\setbox\hlHugeboxopenbrace=\hbox{\begin{Huge}\verb.{.\end{Huge}}%
\setbox\hlHugeboxclosebrace=\hbox{\begin{Huge}\verb.}.\end{Huge}}%
\setbox\hlHugeboxlessthan=\hbox{\begin{Huge}\verb.<.\end{Huge}}%
\setbox\hlHugeboxdollar=\hbox{\begin{Huge}\verb.$.\end{Huge}}%
\setbox\hlHugeboxunderscore=\hbox{\begin{Huge}\verb._.\end{Huge}}%
\setbox\hlHugeboxand=\hbox{\begin{Huge}\verb.&.\end{Huge}}%
\setbox\hlHugeboxhash=\hbox{\begin{Huge}\verb.#.\end{Huge}}%
\setbox\hlHugeboxat=\hbox{\begin{Huge}\verb.@.\end{Huge}}%
\setbox\hlHugeboxbackslash=\hbox{\begin{Huge}\verb.\.\end{Huge}}%
\setbox\hlHugeboxgreaterthan=\hbox{\begin{Huge}\verb.>.\end{Huge}}%
\setbox\hlHugeboxpercent=\hbox{\begin{Huge}\verb.%.\end{Huge}}%
\setbox\hlHugeboxhat=\hbox{\begin{Huge}\verb.^.\end{Huge}}%
\setbox\hlHugeboxsinglequote=\hbox{\begin{Huge}\verb.'.\end{Huge}}%
\setbox\hlHugeboxbacktick=\hbox{\begin{Huge}\verb.`.\end{Huge}}%
\setbox\hlHugeboxhat=\hbox{\begin{Huge}\verb.^.\end{Huge}}%
 

\def\urltilda{\kern -.15em\lower .7ex\hbox{\~{}}\kern .04em}%

\newcommand{\hlstd}[1]{\textcolor[rgb]{0,0,0}{#1}}%
\newcommand{\hlnum}[1]{\textcolor[rgb]{0.16,0.16,1}{#1}}
\newcommand{\hlesc}[1]{\textcolor[rgb]{1,0,1}{#1}}
\newcommand{\hlstr}[1]{\textcolor[rgb]{1,0,0}{#1}}
\newcommand{\hldstr}[1]{\textcolor[rgb]{0.51,0.51,0}{#1}}
\newcommand{\hlslc}[1]{\textcolor[rgb]{0.51,0.51,0.51}{\it{#1}}}
\newcommand{\hlcom}[1]{\textcolor[rgb]{0.51,0.51,0.51}{\it{#1}}}
\newcommand{\hldir}[1]{\textcolor[rgb]{0,0.51,0}{#1}}
\newcommand{\hlsym}[1]{\textcolor[rgb]{0,0,0}{#1}}
\newcommand{\hlline}[1]{\textcolor[rgb]{0.33,0.33,0.33}{#1}}
\newcommand{\hlkwa}[1]{\textcolor[rgb]{0,0,0}{\bf{#1}}}
\newcommand{\hlkwb}[1]{\textcolor[rgb]{0.51,0,0}{#1}}
\newcommand{\hlkwc}[1]{\textcolor[rgb]{0,0,0}{\bf{#1}}}
\newcommand{\hlkwd}[1]{\textcolor[rgb]{0,0,0.51}{#1}}

\newenvironment{Houtput}{\raggedright}{%
%
}
\usepackage[T1]{fontenc}
\usepackage[latin9]{inputenc}
\usepackage{color}
\definecolor{note_fontcolor}{rgb}{0.80078125, 0.80078125, 0.80078125}
\usepackage[authoryear]{natbib}

\makeatletter

%%%%%%%%%%%%%%%%%%%%%%%%%%%%%% LyX specific LaTeX commands.
%% Because html converters don't know tabularnewline
\providecommand{\tabularnewline}{\\}
%% The greyedout annotation environment
\newenvironment{lyxgreyedout}
  {\textcolor{note_fontcolor}\bgroup\ignorespaces}
  {\ignorespacesafterend\egroup}

%%%%%%%%%%%%%%%%%%%%%%%%%%%%%% Textclass specific LaTeX commands.
 %\usepackage{Sweave}
 \ifdefined\Sinput
 \else
 \IfFileExists{Sweave.sty}{
   \usepackage{Sweave}
 }{
   \usepackage{graphicx,fancyvrb}
   \DefineVerbatimEnvironment{Sinput}{Verbatim}{fontshape=sl}
   \DefineVerbatimEnvironment{Soutput}{Verbatim}{}
   \DefineVerbatimEnvironment{Scode}{Verbatim}{fontshape=sl}
   \newenvironment{Schunk}{}{}
   \newcommand{\Sconcordance}[1]{%
     \ifx\pdfoutput\undefined%
       \csname newcount\endcsname\pdfoutput\fi%
       \ifcase\pdfoutput\special{##1}%
     \else%
       \begingroup%
       \pdfcompresslevel=0%
       \immediate\pdfobj stream{##1}%
       \pdfcatalog{/SweaveConcordance \the\pdflastobj\space 0 R}%
       \endgroup%
     \fi}
 }
 \fi
 \usepackage{tikz}
\usetikzlibrary{external}
\tikzexternalize[mode=list and make]


%%%%%%%%%%%%%%%%%%%%%%%%%%%%%% User specified LaTeX commands.
\usepackage{ae}

%% maxwidth is the original width if it's less than linewidth
%% otherwise use linewidth (to make sure the graphics do not exceed the margin)
\def\maxwidth{%
\ifdim\Gin@nat@width>\linewidth
\linewidth
\else
\Gin@nat@width
\fi
}

\usepackage{todonotes}
\usepackage[thickspace]{SIunits}
\usepackage{movie15}

\usepackage{hyperref}
\AtBeginDocument{\renewcommand{\ref}[1]{\mbox{\autoref{#1}}}}

\makeatother

\usepackage{babel}
\begin{document}



\section{Problems}

%setup


%% \maxwidth has been defined in the preamble; see document settings
\setkeys{Gin}{width=\maxwidth}

Here I'm trying to give on a more detailed explanation of the problems
I'm encountering with subject's data in this experiment. Let's pick
apart what's going on.

First a terminological note: one reason why there are so many terms
for local/Fourier/first-order/short-range/phi/.... motion versus ``the
other kind,'' and all the terms seem to be unsatisfactory, is because
they all mix up stimuli with mechanisms\citep{Cavanagh:1991fk}. That
is, people construct experiments with somewhat arbitrarily defined
``first-order'' or ``second order'' motion or ``short-range''
and ``long-range,'' then attempt to infer from that properties of
mechanisms that are imagined to be in correspondence with the stimuli.
I believe there are multiple motion sensing mechanisms at play, but
I would not like to pretend that my stimulus cleanly cleaves them
(and in fact I know that it does not). Therefore I'd like to use terms
that are unambiguously about the stimulus and not the mechanism. Since
the stimuli are Gabor-like, I will talk about their \emph{envelope}
motion as opposed to their \emph{carrier} motion.

The construction of my experiment, currently, is to measure the proportion
of responses clockwise to a stimulus in which either the envelope
or the carrier motion may be either clockwise or counterclockwise. 

One problem is that many subjects, (as in three so far, which exceeds
anyone's good-taste ability to dismiss ``anomalies'') do not actually
show a reversal with this stimulus set. They look more like CF, GB
or AS here.

\begin{figure}[h]
\caption{insert figure of CF, AS or GB here.}
\end{figure}


On the other hand, if you show them the stimuli while chatting with
them -- the exact same stimuli -- they readily agree that something
in the percept fundamentally changes from ``clockwise'' to ``counterclockwise''
when the spacing between targets is reduced below a very reasonable
critical spacing. It just doesn't show up when they answer forced-choice
experiments. %
\begin{lyxgreyedout}
(Maybe the whole ``experiment 3'' exploration of the stimulus space
would work a whole lot faster if I used a method of adjustment, instead
of forced choice + fitting psychometric functions. Method of adjustment
would allow subjects to select the obvious change in quality without
being tricked into forced decisions.)%
\end{lyxgreyedout}


Taking a cue from more recent work on crowding, I set about to look
at thresholds for the two types of motion in my stimulus. In this
pilot experiment the feature that subjects are asked to discriminate
is the envelope motion (really they are asked to discriminate the
appearance of motion, but it is the ways in which the appearance of
motion deviates from the envelope motion that are most interesting,
so we will consider the data in terms of envelope motion.) I have
also looked at some of this data in terms of the discriminating variable
being the carrier motion, but it turns out to be pretty gnarly, as
we may see later on.

So what I will do is to measure psychometric functions that have envelope
motion, parameterized in terms of the \emph{displacement} of the envelope
between appearances. I collect a number of such functions, for various
values of \emph{direction contrast} and\emph{ }inter-element\emph{
spacing.}

Spacing is the distance between adjacent targets on the circle, while
direction contrast is a parameterization of the amount of clockwise
or counterclockwise carrier motion in the stimulus. Thus the ``congruent'',
``counterphase'' and ``incongruent'' motions in my existing experiment
correspond to direction contrasts of 1, 0, and -1. For each combination
of direction contrast and spacing, I run two interleaved staircases,
one 2up-1down, the other 2down-1up, so that I can get a good idea
of both the PSE and the slope of the psychometric function.

%titrate-load


%titrate-process-basic


For reference, here (\ref{fig:data-extent}) are all combinations
of contrast and spacing for which I've collected a displacement staircase.
Here I'm folding trials with clockwise and counterclockwise carrier
motion together, so you only see non-negative values for contrast.

\begin{figure}[h]
%titrate-what-tested


\tikzsetnextfilename{Rfigs/fig-titrate-what-tested}

\tikzexternalfiledependsonfile{Rfigs/fig-titrate-what-tested}{Rfigs/fig-titrate-what-tested.tikz}
\input{Rfigs/fig-titrate-what-tested.tikz}

\caption{\label{fig:data-extent}Current extent of pilot direction discrimination
data. Dots indicate stimulus parameters for which a psychometric function
was collected.}
\end{figure}


Now, as an illustrative example, I'll show my own data for a direction
contrast of 15\%. 

\begin{figure}[h]
%titrate-pbm-example


\tikzsetnextfilename{Rfigs/fig-titrate-pbm-example}

\tikzexternalfiledependsonfile{Rfigs/fig-titrate-pbm-example}{Rfigs/fig-titrate-pbm-example.tikz}
\input{Rfigs/fig-titrate-pbm-example.tikz}

\caption{\label{fig:pbm-example}Example direction discrimination data for
PBM at 15\% carrier motion contrast.}
\end{figure}


Because it seems less confusing, I'll use the terms \emph{left} and
\emph{right} with the understanding that the motions tested are more
properly clockwise and counterclockwise, and moreover the data have
been folded so that each data point represented as having ``rightward''
motion contrast reflects testing with counterclockwise carrier motion
as well as clockwise.

In \ref{fig:pbm-example}, the vertical axis is the probability of
answering ``rightward'' to a stimulus, as a function of the \emph{envelope}
displacement on the horizontal axis; positive values indicate rightward
envelope displacements. Meanwhile, the \emph{carrier} motion of all
stimuli had a contrast of 15\% rightward. A separate psychometric
function is then fit to a wide range of inter-element \emph{spacings.
}At the largest two spacing values there were just two and four elements
on the screen; at the smallest spacing there were 20.

All the functions have a positive slope, that is, more rightward displacement
always makes you more likely to say ``rightward'', which makes sense.
Now, looking at this there are a few things I would point to. One
is that when spacing is on the wide end, the slopes of the psychometric
functions are constant. That is, discriminating the direction of is
just as easy when there are ten objects on screen as when there are
two. This in itself is somewhat interesting; I'm unable to make use
of more targets on the screen even if they are all giving me equally
reliable information.

However the curves to not overlap; the \emph{bias} is not constant.\emph{
}Look at where the curves intersect the vertical line where displacement=0;
adding more targets (reducing spacing) shifts the intercept upwards.
That is, in the uncrowded regime, adding more (rightward-carrier)
targets on the screen makes me more likely to say that they are moving
rightward (without affecting your precision in discriminating direction.)
I'm not sure whether this (change in bias with target density in the
uncrowded regime) is an effect of spacing, total number of objects,
or aggregate motion energy as summed up by an optic flow detector,
but my best guess is the last. 

That addition of rightward carrier motion seems reasonable in this
one example, but note something strange: We can already see one clue
of this weirdness: for the widest spacings, at zero displacement,
I actually tend to answer \emph{leftward, }for a stimulus whose carrier
motion is to the right! Think about this for a second. Put two targets
on the screen, and I answer ``left.'' Show eight and I answer ``right''.
The targets are completely identical and they are too far apart to
be spatially interacting. What's going on?

Moreover the direction of carrier-motion-induced bias is \emph{not}
consistent between subjects, nor even \emph{within} subjects (for
example, bias changes non-monotonically with motion contrast, as I
may show later, or you can read in ) 

These effects are a nuisance and the fact that they happen differently
in different subjects is a further nuisance. What I need is some thoughtful
advice on how to deal with these effects. I would suggest a detailed
reading of \citep{Murakami:1993vn}as well as\citet{Mareschal:2010fk},
since the nuisance effects I'm seeing all seem to be mirrored there.

Now, actual thing in my data which corresponds to ``critical spacing,''
and the qualitative shift in the quality of motion, is seen by looking
at the slopes of the psychometric functions. Something important happens
when spacing reaches a critical value of about 3.5 degrees center-to-center
(yellow-green line); the slopes start to decline, and they decline
further with closer spacings. This is the true measure of spatial
interactions: the spacing at which flanking carrier motions begin
to interfere with the discrimination of envelope motion. The transition
between non-critically spaced and critically spaced is quite clear.

How does this compare to my current experiment? My current experiment
fixes a particular displacement value, and measures response rate
as a function of spacing, with an aim to call the PSE, the point where
the fitted curve crosses 50\%, You can imagine drawing a vertical
line though this graph at a displacement of -0.1 degree; you would
find that as spacing changes from wide to narrow, my responses change
from ``left'' to ``right.'' But this displacement value is arbitrarily
chosen; by selecting several different displacement values, I can
extract several such curves, each having a different PSE (\ref{fig:arbitrary-pse}). 

\begin{figure}


\caption{\label{fig:arbitrary-pse}By choosing different values of displacement
arbitrarily, I can obtain curves that have several different PSEs.
(Imagine this figure, for now.)}


\end{figure}


So we see, the ``PSE'' that I have been measuring as a function
of response rates is not a reliable measure of ``critical spacing.''
It is affected by critical spacing, but also by the subject's individual
pattern of bias in response to carrier motion contrast. It is true
that there is a region of motion contrast and displacement that is
somewhat stable -- for myself and some other subjects, but this is
not generally true for all subjects, and it is probably not true that
changes in the ``PSE'' measured correspond to changes. What holds
true across all subjects is the increase in threshold as a function
of spacing. As \citep{Pelli:2008ao} put the problem,
\begin{quote}
Crowding has usually been characterized by just one number, \textquotedblleft{}critical
spacing\textquotedblright{}, i.e., spacing threshold, the spacing
required to achieve a criterion level of performance. That single
number seems to be enough to characterize crowding when the flanker
is similar to the target, but may not adequately describe the weaker
crowding produced by dissimilar flankers. Disentangling the amplitude
and extent of crowding demands a two-number description. The complete
\textquoteleft{}psychometric function\textquoteright{}, plotting proportion
correct as a function of spacing, tells us little more than the critical
spacing. Proportion correct has a small dynamic range bounded by the
floor at chance, when spacing is below critical, and by the ceiling
at 100\%, when spacing is above critical. To get the whole story,
we must replace proportion correct by a better dependent measure:
threshold. To measure threshold, one varies a physical parameter of
the stimulus to achieve a particular level of performance. Thus, threshold
is measured on a physical scale with a wide dynamic range. For example,
several studies have measured orientation discrimination thresholds
as a function of spacing. These plots show that the weaker crowding
produced by less-similar flankers has much less amplitude (maximum
threshold elevation) but practically the same spatial extent. \citep{Pelli:2008ao}
\end{quote}
In the case of my stimulus I believe we do have ``dissimilar flankers,''
despite the flankers being physically identical to the target; in
crowded conditions it is the motion-energy from the flanker that interferes
with a percept that is normally dominated by the envelope motion,
whereas with a lone target the motion energy is easily captured by
the envelope motion. That is so say, the relevant feature of the target
(envelope motion) is dissimilar to the relevant feature of the flanker
(carrier motion.) In some subjects this effect is indeed weaker than
in other subjects. In fact, for stimuli that put first-order and higher-order
motion systems in conflict, there appears to be a lot of individual
variation, which can be made use of in a between-subjects design\citep{Wilmer:2007lr,Fraser:1979ly}.

But in addition of this need to look at thresholds rather than response
rates, this motion experiment turns out to be more complicated because
the bias is also a function of spacing and motion contrast. So I need
a procedure that extracts this critical spacing that is robust to
the bias effects, and is efficient on top of that.

Let's quantify each psychometric function in terms of slope and bias,
and show the whole of the data.

%titrate-slopes


\begin{figure}
%titrate-slopes-fig


\tikzsetnextfilename{Rfigs/fig-titrate-slopes-fig}

\tikzexternalfiledependsonfile{Rfigs/fig-titrate-slopes-fig}{Rfigs/fig-titrate-slopes-fig.tikz}
\input{Rfigs/fig-titrate-slopes-fig.tikz}

\caption{\label{fig:titrate-slopes}Measured slopes, as a function of target
spacing, grouped by carier contrast.}
\end{figure}


In \ref{fig:titrate-slopes}I have left off the data from carrier
motion contrast of 0, because there is an effect of motion energy
that contaminates the data (see below.) So the data from other subjects
appears limited here, but in inspecting the data from individual sessions
(I can walk you through this), it seems likely that the pattern will
hold.

Also interesting is the pattern of change in bias. In \ref{fig:titrate-bias}we
see that the bias increases with reduced target spacing. For my own
data, this bias appears to increase further once below the critical
spacing, perhaps reflecting ``obligatory summation'' which contributes
to the motion reversal illusion -- but the real driver of motion reversal
is the way that the discriminability for displacement falls off a
cliff.

\begin{figure}
%titrate-bias-fig


\tikzsetnextfilename{Rfigs/fig-titrate-bias-fig}

\tikzexternalfiledependsonfile{Rfigs/fig-titrate-bias-fig}{Rfigs/fig-titrate-bias-fig.tikz}
\input{Rfigs/fig-titrate-bias-fig.tikz}

\caption{\label{fig:titrate-bias}Measured biases, as a function of target
spacing, grouped by carrier contrast.}
\end{figure}



\subsubsection{Good news, bad news}
\begin{itemize}
\item Bad news: Several subjects do not show a ``motion reversal'' when
tested using the current procedure.
\item Good news: You can rescue this with suitable adjustments to displacement
and motion contrast used in testing.
\item Bad news: such adjustments are arbitrary and lead to different PSEs
for different values.
\item Good news: Every subject does show a change in potion processing,
corresponding to the change in subjective appearance to the stimulus
that coincides with the idea of a ``critical distance.''
\item Bad news: The current procedure (measuring response rates as a function
of) does not extract this critical spacing even for subjects where
it successfully extracts a ``PSE.''
\item Good news: I know what the critical spacing means, in much better
detail, and can measure it given enough time.
\item Bad news: I don't know how to measure it efficiently. It's complicated
by biases that vary with motion contrast, in a pattern that is individual
to each subject.
\end{itemize}
\bigskip{}


What's below this line is much more incomplete than what's above.

\rule[0.5ex]{1\columnwidth}{1pt}

It's worth looking at what happens when carrier motion contrast changes.
Things get more complicated here, because our stimulus does not cleanly
separate motion-energy from envelope motion. It turns out that small
envelope displacements contaminate the motion energy of the stimulus
with a couple of percentage points. That did not affect the above
tests at 15\% contrast, but here is some data at 0\% contrast to look
at. There's what looks like non-monotonic psychometric functions. 

\begin{figure}
\caption{Apparently non-monotonic psychometric functions when carrier contrast
is fixed at 0\%.}


\end{figure}


It turns out, when modeling the motion energy contained in each of
these stimuli, that a small rightward displacement of a stimulus with
0\% carrier motion contrast results in some leftward motion energy.

So how to disentangle this? Our measured ``bias'' values are a reflection
of how much perceptual weight is given to motion energy in the stimulus
So we do a multiple logistic regression on displacement and motion
energy?


\section{Motion Energy analysis}

For each unique configuration of the stimulus that was tested, up
to a rotation, we calculated the luminance values along the circle
transecting the element centers, sampled at the monitor frame rate.
To this data we applied a motion-energy model similar to \citep{Adelson:1985ea},
using space-time separable analysis filters. The spatial analysis
filters were intended to approximate the bandpass properties of direction
selective channels in human vision; to that extent we used a set of
Cauchy filters \citep{Klein:1985rz} with the center frequencies matched
to those of the stimuli. The spatial bandwidth of the filters was
dependent on the center frequency and the eccentricity, consistent
with the bandpass properties inferred from human psychophysics; in
particular, at a given eccentricity, the bandwidths of the filters
decrease somewhat with increasing spatial frequency \citep{Anderson:1987oq,Banks:1991kl,Anderson:1991hc}.
We interpolated the measurements of \citet{Banks:1991kl}. The bandwidths
chosen for the analysis filters were thus a function of both eccentricity
and center spatial frequency; we selected the spatial extent of each
analysis filter by interpolating the measurements of . The parameters
used at each eccentricity are given in \ref{tab:Bandwidth}. On the
other hand, the temporal component of each filter was identical at
all cases, because temporal frequency sensitivity does not appear
to vary with spatial frequency or eccentricity \citep{Virsu:1982fv,Wright:1983dz}.
The temporal component of each filter was the same as used in \citet{Kiani:2008uq}
which approximates %
\begin{lyxgreyedout}
something or other (movshon)%
\end{lyxgreyedout}
. 

\begin{table}
\begin{tabular}{|c|c|c|c|}
\hline 
 &
Spatial frequency &
 &
\tabularnewline
\hline 
\hline 
Eccentricity &
 &
 &
\tabularnewline
\hline 
2.96 &
 &
 &
\tabularnewline
\hline 
 &
 &
 &
\tabularnewline
\hline 
\end{tabular}

\caption{\label{tab:Bandwidth}Bandwidth settings used in motion energy analysis
These values were chosen by interpolating measurements performed by
.}
\end{table}


\bibliographystyle{plainnat}
\bibliography{0_Users_peter_analysis_ConcentricDirectionQuest_writing_bibliography}

\begin{quote}
\bibliographystyle{plain}
\bibliography{0_Users_peter_analysis_ConcentricDirectionQuest_writing_bibliography}
\end{quote}

\end{document}
